\RequirePackage{fixltx2e}
\documentclass[a4paper,10pt]{article}
\usepackage[margin=2cm]{geometry}
\usepackage[T1,T2A]{fontenc}
\usepackage[utf8]{inputenx}
\usepackage[british,american,spanish,greek,french,german,polish,english,russian]{babel}
\usepackage{csquotes}
\usepackage{multicol}
\usepackage{fancyvrb}
\usepackage{color}
\definecolor{spot}{rgb}{0,0.2,0.6}
\usepackage[colorlinks,allcolors=spot,bookmarksopen=true,pdfstartview=FitH]{hyperref}
\usepackage[parentracker=true,
  backend=biber,
  hyperref=auto,
  language=auto,
  babel=other,
  citestyle=gost-numeric,
  defernumbers=true,
  bibstyle=gost-footnote,
]{biblatex}
\addbibresource{biblatex-gost-examples.bib}

\tolerance=1500

\renewbibmacro*{begentry}
  {\hspace{-4em}\makebox[4em]{\hyperlink{\thefield{entrykey}}{$\Downarrow$}}%
   \raisebox{\baselineskip}{\hypertarget{back:\thefield{entrykey}}{}}}%
\edef\mytempforAtCharacter{\char64}
\def\getkey#1#2{%
  \csappto{mytempforbibkey}{#1}%
  \ifstrequal{#2}{,}
  {\hspace{-4em}\makebox[4em]{\hyperlink{back:\mytempforbibkey}{$\Uparrow$}}%
   \raisebox{\baselineskip}{\hypertarget{\mytempforbibkey}{}}%
   \mytempforAtCharacter\mytempforbibtype\{\mytempforbibkey,}
  {\getkey#2}%
}
\renewcommand*{\do}[1]{\csdef{#1}##1{\def\mytempforbibtype{#1}\def\mytempforbibkey{}\getkey}}
\docsvlist{MVBOOK,BOOK,INBOOK,SUPPBOOK,BOOKINBOOK,MVCOLLECTION,COLLECTION,INCOLLECTION,SUPPCOLLECTION,MVREFERENCE,REFERENCE,INREFERENCE,PROCEEDINGS,INPROCEEDINGS,PERIODICAL,ARTICLE,PATENT,ONLINE,THESIS,mvbook,book,inbook,suppbook,bookinbook,mvcollection,collection,incollection,suppcollection,mvreference,reference,inreference,proceedings,inproceedings,periodical,article,patent,online,thesis}
\renewcommand*{\do}[1]{\csdef{#1}##1{\mytempforAtCharacter #1\{}}
\docsvlist{XDATA,COMMENT,xdata,comment,}

\title{Biblatex-GOST examples}
\author{Oleg Domanov\thanks{odomanov@yandex.ru}}

\defbibcheck{related}{%
  \iffieldundef{related}{\skipentry}{}}
\defbibcheck{norelated}{%
  \iffieldundef{related}{}{\skipentry}}

\begin{document}
\selectlanguage{english}
\maketitle
Examples are grouped according to their entrytypes. The types themselves are the following
(click to go to the respective bibliography section):
\renewcommand*{\do}[1]{\hyperref[#1]{@#1}\\}
\begin{multicols}{5}
\noindent%
\docsvlist{mvbook,book,inbook,suppbook,bookinbook,mvcollection,collection,incollection,suppcollection,mvreference,reference,inreference,proceedings,inproceedings,periodical,article,patent,online,thesis}
\end{multicols}

See also examples of the \hyperref[related]{\texttt{related}} field.

In the end of the text you may find the listing of the
\hyperref[bibfile]{bibliography database file} used.
You may take advantage of
the arrows on the left to switch between a database entry and its presentation.
The database itself is preliminary. This is actually a file that I use in my everyday work,
so it contains some redundant, provisional and even erroneous (sometimes) material.
I hope this is not going to confuse the reader too much.

\nocite{*}

\selectlanguage{russian}
\defbibheading{bibliography}{}
\renewcommand*{\do}[1]{\section{@#1\label{#1}}\printbibliography[type=#1]}
\docsvlist{mvbook,book,inbook,suppbook,bookinbook,mvcollection,collection,incollection,suppcollection,mvreference,reference,inreference,proceedings,inproceedings,periodical,article,patent,online,thesis}

\renewbibmacro*{begentry}
  {\hspace{-4em}\makebox[4em]{\hyperlink{\thefield{entrykey}}{$\Downarrow$}}}%

\section{\texttt{Related} field\label{related}}
\printbibliography[check=related]

\section{Bibliography database\label{bibfile}}
\VerbatimInput[fontsize=\small,codes={\catcode`\-=11},commandchars=\@\[\]]{biblatex-gost-examples.bib}
\end{document}