% -*- mode: LaTeX; coding: utf-8; fill-column: 90 -*-
\input glyphtounicode.tex           %these three lines are
\input glyphtounicode-cmr.tex       %needed for russian search/copy
\pdfgentounicode=1                  %(cmap)
\documentclass[10pt,a4paper,headings=small,numbers=enddot,english,russian]{ltxdockit}
\usepackage[vscale=0.8,hdivide={0.2\paperwidth,*,0.1\paperwidth}]{geometry}
\usepackage[T1,T2A]{fontenc}
\usepackage[utf8]{inputenc}
\usepackage{btxdockit}
\usepackage{babel}
\usepackage[strict]{csquotes}
\usepackage{microtype}
\usepackage{tabularx}
\usepackage{longtable}
\usepackage{booktabs}
\usepackage{rotating}
\usepackage{comment}
\usepackage[inline]{enumitem}
\usepackage[backend=biber,
bibstyle=gost-standard,
]{biblatex} % to receive \bbx@gost@version, etc.
\usepackage{indentfirst}
\usepackage{paratype}
\usepackage{wrapfig}
\usepackage[labelsep=period,hypcap=true]{caption}
\DeclareCaptionLabelFormat{continued}{#1~#2 (продолжение)}
\setlength{\abovecaptionskip}{0pt}
\setlength{\belowcaptionskip}{1ex}

\lstset{keepspaces=true}

\hypersetup{unicode=true}

\newcommand*{\biber}{Biber\xspace}
\newcommand*{\biblatex}{Biblatex\xspace}
\newcommand*{\biblatexhome}{http://sourceforge.net/projects/biblatex/}
\newcommand*{\biblatexctan}{http://mirror.ctan.org/macros/latex/contrib/biblatex/}
\newcommand*{\biblatexgost}{Biblatex-GOST\xspace}
\newcommand*{\biblatexgosthome}{http://sourceforge.net/projects/biblatexgost/}
\newcommand*{\biblatexgostcode}{https://github.com/odomanov/biblatex-gost/}
\newcommand*{\biblatexgostctan}
    {http://mirror.ctan.org/macros/latex/contrib/biblatex-contrib/biblatex-gost/}

\makeatletter
\usepackage{datetime}
\def\blx@gost@printdateTeX#1/#2/#3//{\formatdate{#3}{#2}{#1}}
\newcommand*{\printdateTeX}[1]{\expandafter\blx@gost@printdateTeX#1//}

\titlepage{%
  title={\biblatexgost},
  subtitle={Оформление библиографии по ГОСТ 7.0.5---2008},
  url={\biblatexgosthome},
  author={Олег Доманов},
  email={odomanov@yandex.ru},
  revision={\bbx@gost@version},
  date={\printdateTeX{\bbx@gost@date}}}

% TOC,LOT layout

\usepackage{tocloft}
\renewcommand*{\cftsecaftersnum}{.}
\renewcommand*{\cftsubsecaftersnum}{.}
\renewcommand*{\cfttabaftersnum}{.}
\setlength{\cftbeforesecskip}{0.25\baselineskip}
\setlength{\cftsecnumwidth}{1.5em}
\cftsetindents{subsection}{\cftsecnumwidth}{1.2\cftsecnumwidth}
\cftsetindents{table}{\cftsecindent}{\cftsecnumwidth}
\renewcommand*{\cftsubsecfont}{\small}
\renewcommand*{\cftsecfont}{\bfseries\sffamily\cftsubsecfont}
\renewcommand*{\cfttabfont}{\cftsubsecfont}
\renewcommand*{\cftsecpagefont}{\cftsubsecfont}
\renewcommand*{\cftsubsecpagefont}{\cftsubsecfont}
\renewcommand*{\cfttabpagefont}{\cfttabfont}

\hypersetup{%
  pdftitle={\biblatexgost Package},
  pdfsubject={Bibliography according to Russian standard GOST 7.0.5-2008},
  pdfauthor={Oleg Domanov},
  pdfkeywords={tex, latex, biblatex, bibliography, references, citation, gost, russian}}

% tables

\newcolumntype{H}{>{\sffamily\bfseries\spotcolor}l}
\newcolumntype{L}{>{\raggedright\let\\=\tabularnewline}p}
\newcolumntype{R}{>{\raggedleft\let\\=\tabularnewline}p}
\newcolumntype{C}{>{\centering\let\\=\tabularnewline}p}
\newcolumntype{V}{>{\raggedright\let\\=\tabularnewline\ttfamily}p}

\setcounter{secnumdepth}{4}

%\hyphenation{}

\definecolor{reddish}{rgb}{0.6,0.2,0.2}
\newcommand*{\reddishcolor}{\color{reddish}}
%----- KOMA-script
\deffootnote[1.0em]{1.0em}{1.0em}{\makebox[0.7em][r]{\textsuperscript{\thefootnotemark}}~}
\addtokomafont{section}{\reddishcolor}
\addtokomafont{subsection}{\reddishcolor}
\addtokomafont{subsubsection}{\reddishcolor}
\addtokomafont{disposition}{\reddishcolor}
\addtokomafont{descriptionlabel}{\spotcolor\rmfamily}
\addtokomafont{caption}{\small}
\renewcommand{\textfraction}{.5}
%-----
\usepackage{ellipsis}% must be loaded after hyperref !!
\newcommand*{\bibsty}{\texttt}
\newcommand*{\gostbibname}[1][]{ГОСТ#1 7.1---2003\xspace}
\newcommand*{\gostcitename}[1][]{ГОСТ#1 7.0.5---2008\xspace}
\newcommand*{\gostheadname}[1][]{ГОСТ#1 7.80---2000\xspace}
\newcommand*{\gostbibref}[2][]{\gostbibname[#1], п.\,#2\xspace}
\newcommand*{\gostciteref}[2][]{\gostcitename[#1], п.\,#2\xspace}
\newcommand*{\gostheadref}[2][]{\gostheadname[#1], п.\,#2\xspace}
\newcommand*{\notimpl}{\footnote{Не реализовано в данной версии \biblatexgost.}}
\excludecomment{notimplc} % not implemented
\renewcommand*{\apxref}{\refs{прил.}{прил.}}
\renewcommand*{\Apxref}{\refs{Прил.}{Прил.}}
\renewcommand*{\tabref}{\refs{табл.}{табл.}}
\renewcommand*{\Tabref}{\refs{Табл.}{Табл.}}
\patchcmd{\refs}{and}{и}{}{}
\newcommand*{\noitemspace}{\vspace{-\itemsep}\vspace{-\parsep}\vspace{-.6ex}}
\newcommand*{\starsdelim}{\centerline{*~~~*~~~*}\nopagebreak}

\tolerance=9999

\newenvironment{bibexample}{\begin{list}
     {}
     {\setlength{\leftmargin}{\parindent}%
      \setlength{\itemindent}{-\leftmargin}%
      %\setlength{\itemsep}{\bibitemsep}%
      \setlength{\parsep}{0pt}}}
  {\end{list}}

\makeatother

\begin{document}

\printtitlepage
\tableofcontents
\listoftables

\section{Введение}
\label{sec:int}

Пакет \biblatexgost{}\footnote{Сайт проекта:
  \url{http://sourceforge.net/projects/biblatexgost/}. Исходный код:
  \url{https://github.com/odomanov/biblatex-gost}.} представляет собой
набор стилей для
\biblatex{}\unspace\fnurl{http://mirror.ctan.org/macros/latex/contrib/biblatex/}
и предназначен для оформления библиографических ссылок в соответствии
с требованиями ГОСТа 7.0.5---2008 (а также 7.1---2003, 7.80---2000,
7.82---2001 и др., см. подробнее \secref{sec:whatfor}).
Пакет будет полезен прежде всего в гуманитарных дисциплинах с высокими
требованиями к оформлению библиографии, многие из которых \bibtex не в
состоянии удовлетворить (это касается в первую очередь многоязычных
библиографий, а также библиографических ссылок в сносках,
использования Ibid., Op. cit. и т.\,д.).

На GitHub можно найти небольшую вики с советами:
\url{https://github.com/odomanov/biblatex-gost/wiki/}.

\subsection{Лицензия}
\label{sec:lic}

Permission is granted to copy, distribute and\slash or modify this software under the
terms of the \lppl, version 1.3c or any later
version\fnurl{http://mirror.ctan.org/macros/latex/base/lppl.txt}.
This package is maintained.

\subsection{Установка}
\label{sec:install}

Пакет входит в \tex~Live, MiK\tex и другие дистрибутивы и может быть установлен
с помощью их менеджеров пакетов. Для установки вручную скачайте архив
\texttt{biblatex-gost-\prm{версия}.tds.zip}, содержащий стандартую структуру дерева TDS.
Распакуйте его в локальное дерево \tex'а и
обновите индексы (в \tex~Live, например, это делается командой \texttt{mktexlsr},
в MiK\tex~--- \texttt{initexmf -{}-update-fndb}). Обратите внимание, что директория,
в которую менеджер пакетов устанавливает пакет в конкретном дистрибутиве,
определяется разработчиками этого дистрибутива.
Эта директория может отличаться (и отличается!) от директории
в вышеупомянутом архиве, поэтому при ручной установке лучше распаковывать
архив в локальное, а не общее дерево \tex'а.

Для работы \biblatexgost версии \makeatletter\bbx@gost@version\makeatother\
требуются, как минимум,
\biblatex~3.8\fnurl{http://sourceforge.net/projects/biblatex/files/biblatex-3.8/}
и \biber~2.8\fnurl{http://sourceforge.net/projects/biblatex-biber/files/biblatex-biber/2.8/}.
%(текущие версии на CTAN).
Пакет может не работать с предыдущими версиями.

\subsection{Важные изменения в этой и некоторых недавних версиях}
\label{sec:critchanges}

Ниже перечислены изменения, на которые стоит обратить внимание с
  точки зрения совместимости с предыдущими версиями. 
Более полный список изменений см. в \Apxref{apx:changelog}.

\subsubsection*{Версия 1.16}

\begin{trivlist}
\item \bibfield{origlanguage} и \bibfield{bookoriglanguage} теперь могут быть списками
    (требуется \biblatex~3.8).  
    Если вы работаете с этими полями прямо, то нужно \cmd{printfield}, \cmd{iffieldundef}
    и пр. заменить на \cmd{printlist}, \cmd{iflistundef} и пр.  
    Если не работаете, то для вас это изменение не важно. 
\end{trivlist}

\subsubsection*{Версия 1.15}

\begin{trivlist}
\item Добавлен пробел перед двоеточием "--- разделителем заголовка и
  подзаголовка (его не было), как требует ГОСТ.  
  Чтобы вернуться к старому поведению (убрать пробел) переопределите:
  \begin{lstlisting}[style=latex]
    \renewcommand*{\subtitlepunct}{\addcolon\space}.
  \end{lstlisting}
\end{trivlist}

\subsubsection*{Версия 1.14}

\begin{trivlist}
\item Изменён вывод авторов в библиографическом списке (см.~\secref{sec:gostbibliography})
  "--- приведён в соответствие с \gostbibname.
\item Изменено сокращение страниц для некоторых языков.  
  См. описание опции \opt{otherlangs}, \secref{sec:newoptions}.
\end{trivlist}

% \subsubsection*{Версия 1.12}

% \begin{trivlist}
% \item Изменения в оформлении диссертаций.  
%   Определены новые поля, некоторые объявлены устаревшими
%   (\foreignlanguage{english}{deprecated}). 
%   См.~\secref{sec:dissers}.
% \item Вернул изменения предыдущей версии, касающиеся заполнении поля \bibfield{science}
%   (устаревшее "--- \bibfield{major}).  
%   Не все отрасли науки требуют слова «наук» (например, не требуют архитектура,
%   искусствоведение, культурология). 
%   См.~\secref{sec:dissers}.
% \item Сдвоенные номера томов теперь разделяются косой чертой, как требует ГОСТ.
%   См. подробнее \secref{sec:newoptions}, опция \opt{doublevols}.
% \end{trivlist}

% \subsubsection*{Версия 1.11a}

% \begin{trivlist}
% \item При заполнении полей \bibfield{major}/\bibfield{speciality} в описании диссертаций
%   теперь нужно писать «экон.», «физ.-мат.» и пр. вместо «экон. наук», «физ.-мат. наук» и
%   пр. См.~\secref{sec:dissers}.
% \end{trivlist}

% \subsubsection*{Версия 1.10}

% \begin{trivlist}
% \item Исправлено сокращение <<док.>> на <<д-ра>>.
% \end{trivlist}

% \subsubsection*{Версия 1.9}

% \begin{trivlist}
% \item Версия не совместима с \biblatex версий ниже 3.5.
% \item В \biblatex~3.5 для работы механизма пропуска даты для ссылок на уникальные работы
%   недостаточно опции \opt{singletitle}, нужно установить дополнительно опцию
%   \opt{uniquebaretitle} или воспользоваться опцией \biblatexgost \opt{dropdates}.
%   Подробнее см.~\secref{sec:newoptions}. 
% \end{trivlist}

% \subsubsection*{Версии 1.8, 1.7, 1.6}

% \begin{trivlist}
% \item Нет критических изменений.
% \end{trivlist}

% \subsubsection*{Версия 1.5}

% \begin{trivlist}
% \item Изменились команды формата имён в
%   заголовках (\secref{sec:headingformat}).
% \end{trivlist}

% \subsubsection*{Версия 1.4}

% \begin{trivlist}
% \item Внутреннее имя списка сокращений изменено на \texttt{shorthand} (было
%   \texttt{shorthands}), в соответсвие с изменением в \texttt{biblatex}.
%   Если вы не используете этот список явно, например, в \cmd{printbiblist}, вас это
%   изменение не затронет.
% \end{trivlist}

% \subsubsection*{Версия 1.1}

% \begin{trivlist}
% \item При \kvopt{movenames}{true} (установлено по умолчанию) `\texttt{and others}' 
%   в поле \bibfield{author/editor}
%   означает, что число авторов/редакторов больше 3-х. Подробнее см.~\secref{sec:newoptions}.
% \end{trivlist}

% \subsubsection*{Версия 0.9.2}

% \begin{trivlist}
% \item Добавлено поле \bibfield{journalcredits}
% \item Удалена настройка языка цитат, поскольку эта функция поддерживается
%   в \biblatex, начиная с версии 2.8a (опция \opt{language}).
% \end{trivlist}

% \subsubsection*{Версия 0.9.1}
% 
% \begin{trivlist}
% \item Переход на \biblatex~2.8 и \biber~1.8
% \end{trivlist}

% \subsubsection*{Версия 0.9}
%
% \begin{trivlist}
% \item Переход на \biblatex~2.7 и \biber~1.7
% \item Опция \opt{related} по умолчанию установлена в \opt{true}, так же как в \biblatex
% \item Обработка поля \bibfield{related} приведена в соответствие с \biblatex,
%   в связи с чем изменилось оформление.
% \item Опция \opt{labelyear} заменена на \opt{labeldate}, в связи с изменениями
%   в \biblatex.
% %\item Исправлена ошибка несовместимости с \texttt{polyglossia}.
% \end{trivlist}
%
% \subsubsection*{Версия 0.8}
%
% \begin{trivlist}
% \item URL в \bibtype{online} теперь выводится всегда, независимо от опций \opt{url}, \opt{biburl},
%   \opt{citeurl}
% \end{trivlist}
%
% \subsubsection*{Версия 0.7.1}
%
% \begin{trivlist}
% \item Изменён вывод URL (в связи с изменениями в \biblatex~2.5)
% \end{trivlist}
%
% \subsubsection*{Версия 0.7}
%
% \begin{trivlist}
% \item Переход на \biblatex~2.4 и \biber~1.4. Пакет не будет работать с предыдущими
%   версиями
% \item Существенно переписаны драйверы \bibtype{thesis} и \bibtype{patent}
%   для оформления диссертаций и патентов, см.~\secref{sec:dissers,sec:patent}
% \item В \bibtype{article} отсутствие поля \bibfield{journaltitle} означает электронную
%   публикацию (не выводится информация об идентифицирующем документе:
%   сведения об ответственности, дата, том, номер и т.\,д.)
% \item Изменён перевод некоторых строк в файле локализации: <<вступительная статья>> вместо
%   <<введение>>, почти все строки, связанные с полем \bibfield{related},
%   оформлением диссертаций и патентов, а также строки,
%   связанные с конкатенацией сведений об ответственности при совпадении имён.
% \item Обратите внимание на проблему, связанную с сортировкой кириллических
%   публикаций (см.~\secref{sec:issues})
% \item Внесены небольшие изменения в оформление записей типа \bibtype{booklet}, \bibtype{misc},
%   \bibtype{online}, \bibtype{periodical}, \bibtype{unpublished}
% \end{trivlist}
%
% \subsubsection*{Версия 0.6}
%
% \begin{trivlist}
% \item Переход на \biblatex~2.1 и \biber~1.1. В основном, это связано с переходом
%   на \texttt{datamodel}.
% \item Поле \bibfield{material} (общее обозначение материала) переименовано в \bibfield{media}.
% \item По умолчанию установлено \kvopt{inbookibid}{false}.
% \item По умолчанию установлено \kvopt{firstinits}{true}.
% \item Изменён перевод на русский язык некоторых строк файла локализации.
% \end{trivlist}

\section{Назначение пакета}
\label{sec:whatfor}

Оформление библиографии регулируется тремя основными ГОСТами:

\newlength{\templ}\settowidth{\templ}{\gostcitename~}
\begin{itemize}[labelwidth=\templ,leftmargin=!,align=left,noitemsep]
  \item[\gostcitename] Библиографическая ссылка.
  \item[\gostbibname] Библиографическая запись. Библиографическое описание.
  \item[\gostheadname] Библиографическая запись. Заголовок.
\end{itemize}
Помимо них используются также стандарты на сокращения, оформление электронных ресурсов и т.\,д.
Пакет \biblatexgost предназначен для оформления библиографических \emph{ссылок} по \gostcitename,
  но не библиографических \emph{записей} по \gostbibname и \gostheadname.
ГОСТ на ссылки использует правила оформления элементов библиографического описания из
  ГОСТов на записи, однако в целом их области применения различаются.
В частности, \gostbibname утверждает о себе самом:

\begin{quotation}
Стандарт распространяется на описание документов, которое составляется библиотеками,
  органами научно-технической информации, центрами государственной библиографии, издателями,
  другими библиографирующими учреждениями.

\emph{Стандарт не распространяется на библиографические ссылки}.
\end{quotation}
То же касается \gostheadname.

Библиографическим списком называется множество библиографических записей с простой структурой.
Если структура сложная, то это множество называется библиографическим указателем
  (терминологию см. в ГОСТ 7.0---99).
Библиографический список или указатель легко спутать со списком ссылок, который можно
  иногда найти в конце книги или статьи.
Действительно, согласно \gostcitename, ссылки могут быть внутри текста, в сноске на той же странице 
  или за текстом (в конце документа или его части).
В последнем случае список ссылок в конце текста \emph{не является библиографическим списком}.
ГОСТ говорит об этом вполне ясно:

\begin{quotation}
Совокупность затекстовых библиографических ссылок не является библиографическим списком или
  указателем, как правило, также помещаемыми после текста документа, и имеющим самостоятельное
  значение в качестве библиографического пособия.
\end{quotation}
В частности, библиографический указатель может содержать не только цитированную, но и, например,
  рекомендуемую литературу.
Хотя оформление ссылок во многом сходно с оформлением записей, имеются следующие отличия
  (см.~\gostciteref{}{4.9, 4.10}):
\begin{itemize}
  \item Заголовок записи в ссылке может содержать имена одного, двух или трех авторов документа.
    \emph{Имена авторов, указанные в заголовке, не повторяют в сведениях об ответственности}
    (согласно \gostheadref{}{5.2}, в заголовке используется только одно имя,
    а согласно \gostbibref{5.2.6.8},
    в сведениях об ответственности обязательно приводят все имена).
  \item Заголовок обязательно применяется в ссылках, содержащих записи на документы созданные
    одним, двумя и тремя авторами.
  \item Допускается знак точку и тире, разделяющие области библиографического описания, заменять
    точкой.
  \item Допускается не использовать квадратные скобки для сведений, заимствованных не из
    предписанного источника информации.
  \item Сокращение отдельных слов и словосочетаний применяют для всех элементов библиографической
    ссылки, за исключением основного заглавия документа
    (при оформлении по \gostbibname не сокращаются никакие заглавия).
  \item В области физической характеристики указывают либо общий объем документа, либо сведения о
    местоположении объекта ссылки в документе (при оформлении по \gostbibname указывается
    физическая форма объекта, его размеры, сопроводительные материалы и пр.).
\end{itemize}

Кроме того, хотя об этом различии нигде специально не говорится,
  в \gostheadname инициалы в заголовке отделяются от фамилии запятой с пробелом, а в
  \gostcitename~--- только пробелом.
К сожалению, все эти различия не так уж малы.
Например, при использовании имени редактора в качестве заголовка записи оно должно употребляться
  в именительном падеже, но при этом одновременно присутствовать в сведениях
  об ответственности, где требуется родительный падеж (под. ред. \textellipsis).
В \latex это непросто реализовать.

В итоге, данный пакет \emph{не предназначен} для оформления библиографических списков или
  указателей
  (об ограниченной поддержке этой функции см.~\secref{sec:gostbibliography}).
Он ориентируется на авторов текстов, а не на библиографирующие учреждения.

\section{Стили \biblatexgost}
\label{sec:styles}

ГОСТ определяет три типа ссылок: внутритекстовую, подстрочную и затекстовую.
Первые два типа приблизительно соответствуют стилям \bibsty{verbose} в \biblatex
с опцией \opt{autocite}, установленной, соответственно, в \opt{inline} и
\opt{footnote}. Затекстовые ссылки делятся, в свою очередь, на три типа, в зависимости
от способа отсылки к библиографии (которая в этом случае располагается
\enquote{за текстом}~--- в конце статьи, главы, книги и пр.).
В первом случае используются номера в квадратных скобках, во
втором~--- автор и год в квадратных скобках, в третьем~--- номера в верхнем индексе.
В \biblatex им приблизительно
соответствуют стили \bibsty{numeric} и \bibsty{authoryear}. Таким образом,
стили \biblatexgost являются модификациями соответствующих базовых стилей
\biblatex или даже основываются на них.

Более конкретно, \biblatexgost содержит следующие стили.

\begin{marglist}

\item[gost-inline]
Внутритекстовые ссылки. Ссылки помещаются внутри текста в круглых скобках.
Стиль отслеживает
повторные ссылки, используя при необходимости короткие названия из поля
\bibfield{shorttitle} (о многоточиях при сокращении заголовков
см.~\secref{ellipsis}).

По умолчанию стиль устанавливает опции \kvopt{autocite}{inline}, \kvopt{sorting}{ntvy},
 \kvopt{pagetracker}{true}, \kvopt{strict}{true}, \kvopt{citetracker}{constrict},
\kvopt{opcittracker}{constrict}, \kvopt{citepages}{omit},
\kvopt{labelyear}{true}.

\item[gost-footnote]
Подстрочные ссылки.
Стиль предназначен для помещения ссылок в сносках внизу страницы (например, с помощью
команд \cmd{footcite}, \cmd{smartcite} или \cmd{autocite}).

Фактически, стиль \bibsty{gost-footnote} имеет только два отличия от стиля
\bibsty{gost-inline}.
Во-первых, он устанавливает опцию \kvopt{autocite}{footnote}, что заставляет команду
\cmd{autocite} всегда выводить цитату в сноске
(См.~\secref{sec:citecommands}). Во-вторых, в согласии с требованиями ГОСТа,
во внутритекстовой ссылке (но не
в библиографии!) не выводятся серия и сведения, относящиеся
к заглавию (поле \bibfield{titleaddon}, см.~\tabref{tab:gost-biblatex}). Кроме того,
там не выводятся поля \bibfield{doi},
\bibfield{eprint}, \bibfield{url}, \bibfield{addendum}, \bibfield{pubstate}.

По умолчанию стиль устанавливает опции \kvopt{autocite}{footnote}, \kvopt{sorting}{ntvy},
\kvopt{pagetracker}{true}, \kvopt{strict}{true}, \kvopt{citetracker}{constrict},
\kvopt{opcittracker}{constrict}, \kvopt{citepages}{omit},
\kvopt{labelyear}{true}.

\item[gost-numeric]
Затекстовые ссылки. Основан на стандартном стиле \bibsty{numeric-comp} и отличается
от него лишь оформлением библиографии, но не цитат. Ссылки помещаются после текста,
а для их связи с текстом используются числа в квадратных скобках. Стиль можно использовать
в качестве только
библиографического, указывая как стили цитирования стандартные стили \bibsty{numeric},
\bibsty{numeric-comp} и \bibsty{numeric-verb} и переопределив символ разделения ссылок на
точку с запятой:

\begin{lstlisting}[style=latex]
\renewcommand*{\multicitedelim}{\addsemicolon\space}
\end{lstlisting}

По умолчанию стиль устанавливает опции \kvopt{autocite}{inline}, \kvopt{sortcites}{true}.
\kvopt{labelnumber}{true}.

\item[gost-authoryear]
Затекстовые ссылки. Основан на стандартном
стиле \bibsty{authoryear-icomp}. Ссылки помещаются после текста, и для их связи
с текстом используются имя автора и год в квадратных скобках. 
При отсутствии автора
используется название или поле \bibfield{shorttitle}, если оно доступно.
При наличии поля \bibfield{heading} используется оно, даже если одновременно присутствуют
автор, редактор или переводчик. Нужно заметить, что ГОСТ
не слишком строго определяет соответствующий стиль и, в частности, допускает неопределённые
ссылки в случаях, когда один автор имеет несколько
работ одного года. Поэтому в \biblatexgost используется несколько способов оформления
ссылок, которые выбираются опцией \opt{mergedate} (см.~\secref{sec:newoptions}).
В частности, значение опции \kvopt{mergedate}{goststrict} соответствует строгому
следованию ГОСТу, хотя при этом могут появляться неопределённые ссылки.
  Стиль чувствителен к опции \opt{dropdates} и может не выводить
  год в ссылке, если он не требуется для устранения неопределённости (например, если
  имеется лишь одна публикация данного автора или лишь одна публикация без автора с данным
  названием).

По умолчанию стиль устанавливает опции
\kvopt{autocite}{inline},
\kvopt{sorting}{nyt},
\kvopt{pagetracker}{true},
\kvopt{mergedate}{gostletter},
\kvopt{dropdates}{false},
(или \opt{true} при \kvopt{mergedate}{goststrict},
см.~\secref{sec:newoptions}).

\item[gost-alphabetic]
Этот стиль не предусмотрен ГОСТом, но добавлен для полноты. Он соответствует стандартному
стилю \bibsty{alphabetic}, а также \bibsty{alphabetic-verb}, если его использовать в виде:

\begin{ltxcode}
\usepackage[%
    citestyle=alphabetic-verb,
    bibstyle=gost-alphabetic,
    ...
]{biblatex}
\end{ltxcode}

По умолчанию стиль устанавливает опции
\kvopt{autocite}{inline},
\kvopt{labelalpha}{true},
\kvopt{sorting}{anyvt},

\item[gost-inline-min]%
\item[gost-footnote-min]\vspace{-\itemsep}\vspace{-\parsep}%
\item[gost-numeric-min]\vspace{-\itemsep}\vspace{-\parsep}%
\item[gost-authoryear-min]\vspace{-\itemsep}\vspace{-\parsep}%
\item[gost-alphabetic-min]\vspace{-\itemsep}\vspace{-\parsep}%
\item\vspace{-\itemsep}\vspace{-\parsep}%
\vspace{-4\baselineskip}%
Все стили имеют <<минимальный>> вариант, в котором выводится минимальное количество
сведений~--- автор, название, том\slash часть\slash книга\slash выпуск,
выходные данные, страницы (для статей в журналах и книгах).
Сведения об ответственности, издании, серии и т.\,д. не выводятся, вне зависимости от их присутствия в базе данных.
При этом вывод сведений, для которых в \biblatex имеются специальные опции
(\opt{url}, \opt{doi}, \opt{isbn/issn/isrn}, \opt{eprint}), регулируется
этими опциями (а также опциями \opt{cite\ldots} и \opt{bib\ldots},
описанными в \secref{sec:newoptions}). По умолчанию в минимальных стилях этот вывод отключён.

Минимальные стили можно использовать совместно с полными.
Например, при установке

\begin{ltxcode}
\usepackage[%
    citestyle=gost-footnote-min,
    bibstyle=gost-footnote,
    ...
]{biblatex}
\end{ltxcode}

цитаты будут выводится в минимальном стиле, а записи в библиографии~--- в полном.
Разумеется, это не имеет значения для стилей \bibsty{gost-numeric-min} и
\bibsty{gost-authoryear-min}, которые <<минимизируют>> оформление библиографии, но не цитат.

Кроме того, если вы для цитат используете полный стиль, но одну из них вам нужно привести
в минимальном, поставьте перед ней

\begin{ltxcode}
\AtNextCite{\renewbibmacro*{cite:clearfields}{\usebibmacro{setup:min}}}.
\end{ltxcode}

\end{marglist}

\starsdelim

Дополнительно, все стили по умолчанию устанавливают опции
\kvopt{dashed}{false},
\kvopt{useeditor}{false},
\kvopt{usetranslator}{false},
\kvopt{maxnames}{3},
\kvopt{minnames}{1}.
Значения по умолчанию  для опций \opt{cite\ldots} и \opt{bib\ldots} см. в \tabref{tab:bibcite}.

\vspace{.5\baselineskip}
\starsdelim

ГОСТ определяет ещё один тип затекстовых ссылок, при котором они
связываются с текстом номером в верхнем индексе. Этого можно достичь несколькими способами.

\begin{itemize}
\item Использовать стиль \bibsty{gost-numeric} и команду цитирования
\cmd{supercite}, доступную только для стилей типа \bibsty{numeric}. Эта команда
помещает номер в верхний индекс вместо квадратных скобок. Эти номера не зависят от
сносок \cmd{footnote} и существуют параллельно с ними.
\item Использовать стиль \bibsty{gost-numeric} и команду цитирования
\cmd{autocite}. При установке опции \kvopt{autocite}{superscript} эта команда ведёт себя
как \cmd{supercite}.
\item Использовать стиль \bibsty{gost-footnote} и какой-либо пакет, поддерживающий команду \cmd{endnote} (например, \sty{endnotes}, \sty{memoir} и т.\,д.).
При этом, чтобы ссылки переносились в конец текста (или главы, или в любое
нужное место), должна быть установлена
опция \kvopt{notetype}{endonly}. Можно также воспользоваться
пакетом \sty{fn2end} или ему подобными.
\end{itemize}

\section{Заполнение базы данных}
\label{sec:database}


\subsection%
%[Термины ГОСТа и поля Bib\-\latex\kern-0.3em\nohyphenation-\\GOST]
{Термины ГОСТа и поля \biblatexgost}
\label{sec:gost-biblatex}

Соответствие между терминами ГОСТа и полями \biblatexgost представлено в
\tabref{tab:gost-biblatex}.
Как видно, в ней отсутствуют, с одной стороны, некоторые элементы библиографического
описания, предусмотренные ГОСТом, а с другой~--- некоторые поля \biblatex. Эти элементы и поля
не учитываются в \biblatexgost и не обрабатываются (разумеется, специальные поля,
такие как \bibfield{crossref}, \bibfield{options} и пр., учитываются, хотя и не
представлены в таблице).
Курсивом выделены факультативные, согласно ГОСТу, элементы.
Цветом выделены поля \biblatexgost, отсутствующие в \biblatex
(см.~\secref{sec:newfields,sec:patent}).
В угловые скобки помещены поля, используемые не вполне стандартными типами записей,
такими как \bibtype{patent} или \bibtype{thesis} (см.~\secref{sec:dissers,sec:patent}).

\begingroup
\tablesetup
%\setlength\LTleft{0pt}
%\setlength\LTright{0pt}
\begin{longtable}[l]{@{}L{0.6\textwidth}@{}V{0.4\textwidth}@{}}
\caption{Соответствие между терминами ГОСТа и полями \biblatexgost\label{tab:gost-biblatex}} \\
\toprule
\multicolumn{1}{@{}H}{ГОСТ} &
\multicolumn{1}{@{}H}{\biblatexgost\hfill}  \\
\cmidrule(r){1-1}\cmidrule{2-2}
\endfirsthead
\captionsetup{labelformat=continued}
\caption[]{Соответствие между терминами ГОСТа и полями \biblatexgost} \\
\toprule
\multicolumn{1}{@{}H}{ГОСТ} &
\multicolumn{1}{@{}H}{\biblatexgost\hfill}  \\
\cmidrule(r){1-1}\cmidrule{2-2}
\endhead
\bottomrule
\endfoot
\endlastfoot
Заголовок                        & author, editor, translator, {\spotcolor heading} \\
\multicolumn{1}{@{}H}{Область заглавия и сведений об ответственности} & \\*
Основное заглавие                & title, maintitle, booktitle \\
\textit{Общее обозначение материала} &  {\spotcolor media} \\
\textit{Параллельное заглавие} &  \textendash \\%{\spotcolor paratitle} \\
\textit{Сведения, относящиеся к заглавию} & subtitle, mainsubtitle,
                                    booksubtitle, titleaddon, maintitleaddon,
                                    booktitleaddon, type,\newline
                                    \ensuremath\langle{\spotcolor ipc}, {\spotcolor location},
                                    {\spotcolor number},
                                    {\spotcolor science}, {\spotcolor specialitycode},
                                    {\spotcolor speciality}\ensuremath\rangle \\
Сведения об ответственности      & {\spotcolor credits}, {\spotcolor bookcredits},
                                    editor, editora, editorb, editorc, afterword,
                                    annotator, commentator, foreword, introduction, organization,
                                    translator,
                                    {\spotcolor bookafterword}, {\spotcolor bookannotator},
                                    {\spotcolor bookcommentator}, {\spotcolor bookforeword},
                                    {\spotcolor bookintroduction}, {\spotcolor booktranslator} \\
\cmidrule(r){1-1}\cmidrule{2-2}
\multicolumn{1}{@{}H}{Область издания} & \\*
Сведения об издании & edition, version, {\spotcolor editioncredits} \\
\textit{Параллельные сведения об издании} & edition \\
\cmidrule(r){1-1}\cmidrule{2-2}
\multicolumn{1}{@{}H}{Область специфических сведений} & {\spotcolor specdata},\newline
                          \ensuremath\langle{\spotcolor reqnumber}, {\spotcolor date},
                          {\spotcolor publication}\ensuremath\rangle \\
\cmidrule(r){1-1}\cmidrule{2-2}
\multicolumn{1}{@{}H}{Область выходных данных} & \\*
Место издания, распространения & location \\
Имя издателя, распространителя и т.\,п. & publisher, institution \\
Дата издания, распространения и т.\,п. & date \\
\cmidrule(r){1-1}\cmidrule{2-2}
\multicolumn{1}{@{}H}{Область физической характеристики} & \\*
Специфическое обозначение материала и объём & volumes, {\spotcolor books}, {\spotcolor parts},
                                              {\spotcolor issues}, pagetotal, pages,
                                              howpublished \\
\textit{Другие сведения о физической характеристике} & \textendash \\
\textit{Размеры} & \textendash \\
\textit{Сведения о сопроводительном материале} & \textendash \\
\cmidrule(r){1-1}\cmidrule{2-2}
\multicolumn{1}{@{}H}{Область серии} & \\*
Основное заглавие серии или подсерии & series \\
\textit{Параллельное заглавие серии или подсерии} & series \\
\textit{Сведения, относящиеся к заглавию серии или подсерии} & series \\
Сведения об ответственности, относящиеся к заглавию серии или подсерии & series \\
ISSN серии или подсерии & issn (isbn) \\
Номер выпуска серии или подсерии & number \\
\cmidrule(r){1-1}\cmidrule{2-2}
\multicolumn{1}{@{}H}{Область примечания} & url, urldate, note, type,\newline
                                            \ensuremath\langle{\spotcolor update},
                                            {\spotcolor systemreq}\ensuremath\rangle\\
\cmidrule(r){1-1}\cmidrule{2-2}
\multicolumn{1}{@{}H}{Область стандартного номера и} & \\
\multicolumn{1}{@{}H}{условий доступности} & \\*
Стандартный номер (или его альтернатива) & isbn, issn, %isan, ismn, isrn, iswc,
                           doi, eprint, eprintclass, addendum,\newline
                           \ensuremath\langle{\spotcolor prnumber}, {\spotcolor prdate},
                           {\spotcolor prcountry}\ensuremath\rangle \\
Ключевое заглавие & addendum  \\
Условия доступности и (или) цена & addendum \\
Дополнительные сведения к элементам области & addendum \\
\bottomrule
\end{longtable}
\endgroup

Таким образом, не учитываются следующие поля \biblatex:
%\bibfield{pagination},
%\bibfield{bookpagination},
\bibfield{chapter}, \bibfield{library},
\bibfield{origdate}, \bibfield{origlocation}, \bibfield{origpublisher}, \bibfield{origtitle},
\bibfield{file}, \bibfield{reprinttitle}.

С другой стороны, не учитываются следующие элементы описания, предусмотренные ГОСТом:
параллельное заглавие, другие сведения о физической характеристике, размеры,
сведения о сопроводительном материале.

\subsection{Новые поля}
\label{sec:newfields}

В \biblatexgost определены следующие
поля библиографической базы данных, отсутствующие в стандартном \biblatex
(см. также поля для оформления диссертаций и авторефератов в \secref{sec:dissers},
а также патентов в \secref{sec:patent}).

\begin{fieldlist}

\fielditem{book}{number/literal}

Содержит номер (или другой идентификатор) книги для изданий, разделённых на книги. См. подробнее
\secref{sec:volsparts}

\fielditem{booktranslator}{name}\noitemspace%
\fielditem{bookcommentator}{name}\noitemspace%
\fielditem{bookannotator}{name}\noitemspace%
\fielditem{bookintroduction}{name}\noitemspace%
\fielditem{bookforeword}{name}\noitemspace%
\fielditem{bookafterword}{name}\noitemspace%
\fielditem{bookoriglanguage}{key}\noitemspace%
\listitem{bookcredits}{literal}\noitemspace%
\listitem{journalcredits}{literal}

Аналогично \bibfield{bookauthor} и \bibfield{booktitle}, эти поля
  содержать переводчика, автора комментариев и т.\,д. документа, в состав которого входит
  данная публикация (ГОСТ называет его идентифицирующим документом).
Эти поля определены для записей типа
  \bibtype{inbook}, \bibtype{bookinbook}, \bibtype{suppbook}, \bibtype{incollection},
  \bibtype{suppcollection}, \bibtype{inproceedings}
  и \bibtype{inreference}
  (за исключением \bibfield{journalcredits}, которое определено для \bibtype{article}).
При использовании ссылок \texttt{crossref} или \texttt{xref} эти поля копируются из записи,
  на которую указывает ссылка.

Данные поля могут использоваться, например, при описании сборников переводов классических текстов,
каждый из которых имеет собственного переводчика, комментатора, автора вступительной статьи
и т.\,д. В библиографии, если поле \bibfield{commentator}, \bibfield{annotator} и т.\,д.
совпадает с соответствующим полем \bibfield{bookcommentator}, \bibfield{bookannotator} и т.\,д.,
поле выводится только в описании идентифицирующего документа. В случае переводчика проверяется
ещё и равенство \bibfield{origlanguage} и \bibfield{bookoriglanguage}. Например,
следующие данные

\begin{lstlisting}[style=bibtex,escapechar=|]
@COLLECTION{coll,
  title = {|Сборник|},
  translator = {|А. Б. Ивановой|},
  origlanguage = {finnish},
  ...
}

@INCOLLECTION{art,
  crossref = {coll},
  title = {|Статья|},
  translator = {|А. Б. Ивановой|},
  origlanguage = {finnish},
  ...
}
\end{lstlisting}

выводятся как:

\begin{bibexample}
\item Статья // Сборник / пер. с фин. А. Б. Ивановой ; \ldots
\end{bibexample}

но не как:

\begin{bibexample}
\item Статья / пер. с фин. А. Б. Ивановой // Сборник / пер. с фин. А. Б. Ивановой ; \ldots
\end{bibexample}

Поле \bibfield{bookoriglanguage}, так же как и поле \biblatex \bibfield{origlanguage},
может содержать только один язык. Поэтому, если вам требуется указать список языков,
используйте поля \bibfield{bookcredits} и \bibfield{credits} (см. ниже).

\fielditem{books, parts, issues}{number/literal}

Содержат количество книг, частей, выпусков изданий,
которые делятся на книги, части, выпуски (аналогично стандартному полю \bibfield{volumes}). См.
подробнее \secref{sec:volsparts}.

\listitem{credits}{literal}

Содержит сведения об ответственности, отличные от предусмотренных в \biblatex,
такие, например, как организация:

\begin{bibexample}
\item \textellipsis\ \slash\ Академия наук СССР\ ;\ \textellipsis
\end{bibexample}

В этом поле могут содержаться любые нестандартные сведения об ответственности,
например: <<авт. курса А. Сигалов>>,
<<коллектив авт. под рук. Исакова Ю.Ф>> и т.\,д.
Более того, в особенно нестандартных ситуациях
оно может содержать вообще все сведения об ответственности.
Поле \bibfield{credits} выводится в сведениях об ответственности в самом начале или
сразу после авторов, если они там присутствуют.

\listitem{editioncredits}{literal}

Сведения об ответственности издания, если они относятся к конкретному изданию.
Выводятся сразу после сведений об издании (см.~\gostbibref{5.3.3}).

\fielditem{editortype}{key}

Это поле существует в \biblatex, но в \biblatexgost для него определёно два новых
значения:
\begin{itemize}
\item \bibfield{geneditor} "--- выводится в виде: \enquote{под общ. ред.~\ldots}.
\item \bibfield{editorcollaborator} "--- выводится в виде: \enquote{при ред. уч.~\ldots}.
\end{itemize}

Обратите внимание, что в \biblatex есть также редакторская роль \bibfield{collaborator},
которая выводится как \enquote{при уч.~\ldots}.

\listitem{heading}{name}

Заголовок библиографической записи.

Согласно \gostheadname <<Библиографическая запись. Заголовок>>,
заголовок состоит из основной части и идентифицирующих признаков.
Первая может включать: <<имя лица, наименование организации, унифицированное
заглавие, обозначение документа, географическое название и т.\,д.>>,
в качестве вторых могут быть приведены: <<даты,
специальность, титул, сан, номер, название местности н другие сведения>>.
В \biblatex нет понятия заголовка, однако фактически поля \bibfield{author}, \bibfield{editor} и
\bibfield{translator} играют роль того, что ГОСТ называет заголовком. Можно сказать,
что в \biblatex реализован лишь один тип заголовка~--- имя лица. Поле \bibfield{heading}
позволяет определять заголовки других типов.%

Примеры использования заголовков см. в \secref{sec:patent,sec:standards}.

О формате вывода заголовков см. \secref{sec:headingformat}.

\fielditem{media}{literal}

Общее обозначение материала (\gostbibref{5.2.3}). В \tabref{tab:media} приведены
допустимые значения этого поля вместе с соответствующими терминами ГОСТа.
%
\begin{table}[htbp]
\tablesetup
\centering
% \vspace{-\baselineskip}
\caption{Общее обозначение материала\label{tab:media}}
% \medskip
\begin{tabularx}{.7\textwidth}{@{}V{.4\textwidth}@{}L{.3\textwidth}@{}}
\toprule
\multicolumn{1}{@{}H}{Значение поля \bibfield{media}} &
\multicolumn{1}{@{}H}{Термины ГОСТа}  \\
\cmidrule(r){1-1}\cmidrule{2-2}
videorecording & видеозапись \\
soundrecording & звукозапись \\
graphic & изоматериал \\
cartographic & карты \\
kit & комплект\\
motionpicture & кинофильм\\
microform & микроформа\\
multimedia & мультимедиа\\
music & ноты\\
object & предмет\\
manuscript & рукопись\\
text & текст\\
braille & шрифт Брайля\\
eresource & электронный ресурс\\
\bottomrule
\end{tabularx}
\end{table}
%
Для этих значений определены строки в файле локализации, то есть они могут быть локализованы.
Язык вывода поля
определяется командой \cmd{gostmedialanguage}. По умолчанию он совпадает с
основным языком \texttt{babel}, если последний загружен. Если \texttt{babel} не загружен,
то язык установлен в \texttt{russian}. При переопределении

\begin{ltxcode}
\renewcommand*{\gostmedialanguage}{}
\end{ltxcode}

используется текущий язык библиографической записи или цитаты.
Это важно для многоязычных библиографий, если установлена опция
\kvopt{autolang}{other} или \kvopt{autolang}{other*}, и записи выводятся на разных языках.

\fielditem{sortvolume}{literal}

Значение, используемое вместо тома для сортировки.
Аналогично стандартным
полям \bibfield{sortname}, \bibfield{sorttitle}, \bibfield{sortyear}.
Может потребоваться для настройки порядка вывода томов многотомных изданий
(см.~\secref{sec:volsparts}).

\listitem{specdata}{literal}

Область специфических сведений (см.~\tabref{tab:gost-biblatex}). Представляет собой список,
элементы которого при выводе разделяются точкой с запятой.

\fielditem{systemreq}{literal}

Системные требования для записей типа \bibtype{online}.

\fielditem{update}{date}

Дата обновления для записей типа \bibtype{online}.

\fielditem{volsorder}{literal}

Порядок вывода томов в многотомных изданиях. См.~\secref{sec:volsparts}.

\end{fieldlist}

\subsection{Особенности размещения отдельных сведений}
\label{sec:db:inf}

Б\'{о}льшая часть полей и сведений, таких как автор, название, редактор и т.\,д., оформляются
автоматически.
Ниже приведены рекомендации по размещению сведений, для которых отсутствуют
специальные поля в базе данных или нет прямого соответствия между ГОСТом и \biblatexgost.

\begin{description}
%\item[Общее обозначение материала и параллельное заглавие]:
%
%\begin{lstlisting}[style=bibtex,escapechar=|]
%title = {Google |[Электронный ресурс]|}
%maintitle = {|Аlbumlapok [Ноты] = Аlbumblatter = Аlbum-leaves|}
%\end{lstlisting}
%
\item[Область специфических сведений] применяется для
особых типов публикации и специфических носителей (см.~\gostbibref{5.4}).
\biblatexgost заполняет эту область самостоятельно только для
патентов.
%и сериальных изданий\notimpl.
Для всех остальных типов публикаций
следует помещать информацию в поле \bibfield{specdata}.

%В случае сериальных изданий эта область является областью нумерации, в которой приводятся
%первый и последний номер, перерывы в нумерации и пр. \biblatexgost оформляет это поле
%автоматически, если присутствует поле \bibfield{date} в записях типа \bibtype{periodical}\notimpl.

\item[Cведения об ответственности серии] в случае необходимости можно помещать
в поле \bibfield{series}:
\begin{lstlisting}[style=bibtex,escapechar=|]
series = {|Языковеды мира / редкол.: Г.В. Степанов (пред.) и др.|}
\end{lstlisting}

\item[Серия.] \label{series}ГОСТ различает (см. \gostbibref{6.3.4.2}) сериальные документы
с нумерацией по годам
(как, например, в журналах) и со сквозной нумерацией. В первом случае, кроме прочего,
не печатается год на месте выходных данных. В \biblatexgost только документы
с нумерацией по годам соответствуют \bibtype{periodical}.  Сериальные документы со
сквозной нумерацией следует оформлять как \bibtype{mvbook}, \bibtype{mvcollection},
\bibtype{mvproceedings} или \bibtype{mvreference}.

Серия в журналах может пониматься в двух смыслах. Во-первых, она употребляется в
том же смысле, что и для книг, и также выводится в области серии в виде:

\begin{bibexample}
\item \textellipsis~"--- (Формальная семантика ; 78).~"--- \textellipsis
\end{bibexample}

При заполнении базы данных, эта серия помещается в поле \bibfield{series}.

В случае двух серий, они должны выводиться последовательно, каждая в круглых скобках, и
разделяться пробелом (см.~\gostbibref{5.7.16}).
Этого можно добиться следующим образом:

\begin{lstlisting}[style=bibtex,escapechar=|]
series = {|Полное собрание сочинений : для фортепиано ; т. 1) (Классика мировой музыки|},
\end{lstlisting}

Во-вторых, серия может входить в заглавие журнала:

\begin{bibexample}
\item Известия Российской академии наук. Серия геологическая
\item Труды исторического факультета МГУ. Серия 4, Библиографии
\item Вестник Ивановского государственного университета. Серия «Химия, биология»
\item Итоги науки и техники. Серия: Автомобилестроение
\end{bibexample}

При заполнении базы данных, эта серия помещается в поле \bibfield{journaltitle}
вместе с названием журнала:

\begin{lstlisting}[style=bibtex,escapechar=|]
@ARTICLE{article,
  ...
  journaltitle = {|Итоги науки и техники. Серия: Автомобилестроение|},
}
\end{lstlisting}

\item[Области примечания, стандартного номера и условий доступности.] В случае
необходимости, вся эта информация включается в \bibfield{addendum}. Нужно лишь учесть, что
в стиле \bibsty{gost-inline} она не выводится в цитатах (но выводится в библиографии).

\end{description}

\subsection{Особенности заполнения отдельных полей}
\label{sec:filling}

\begin{fieldlist}

\listitem{afterword}{name}\noitemspace%
\listitem{annotator}{name}\noitemspace%
\listitem{commentator}{name}\noitemspace%
\listitem{foreword}{name}\noitemspace%
\listitem{introduction}{name}

Это поля должны быть в родительном падеже. Форма их вывода (в полной и
краткой форме) представлена в~\tabref{tab:afterword}.

\begin{table}[htbp]
\tablesetup
\centering
\caption{Вывод полей \bibfield{afterword} и пр.\label{tab:afterword}}
\begin{tabularx}{.8\textwidth}{@{}V{.2\textwidth}@{}L{.3\textwidth}@{}L{.3\textwidth}@{}}
\toprule
\multicolumn{1}{@{}H}{Поле} &
\multicolumn{2}{@{}H}{Печатается как}  \\
\cmidrule(r){1-1}\cmidrule{2-3}
afterword & послесловие \prm{кого} & послесл. \prm{кого} \\
annotator & примечание \prm{кого} & примеч. \prm{кого} \\
commentator & комментарий \prm{кого} & коммент. \prm{кого} \\
foreword & предисловие \prm{кого} & предисл. \prm{кого} \\
introduction & вступительная статья \prm{кого} & вступит. статья \prm{кого} \\
\bottomrule
\end{tabularx}
\end{table}

\fielditem{date}{date}

Информация о дате в публикациях типа \bibtype{article} выводится в следующем порядке
(см.~\gostbibref{7.5.2}):

\begin{bibexample}
\item \prm{year}.~"---
\cmd{mkbibdatelong\{\}\{month\}\{day\}}%
.~"--- \prm{volume}, \prm{issue}, \prm{number}
\end{bibexample}

Таким образом, для газет получается (на русском языке):

\begin{bibexample}
\item\textellipsis\ 2001.~"--- 25 мая.~"--- \textellipsis
\end{bibexample}

И для журналов:

\begin{bibexample}
\item\textellipsis\ 2001.~"--- №~1.~"--- \textellipsis
\item\textellipsis\ 2001.~"--- Т.~17.~"--- \textellipsis
\item\textellipsis\ 2001.~"--- Vol.~34, Summer, No.2.~"--- \textellipsis
\item\textellipsis\ 2001.~"--- July.~"--- Vol.~34.~"--- \textellipsis
\end{bibexample}

См.~также \secref{sec:dates} о предполагаемых и открытых датах.

\fielditem{edition}{integer/literal}

Сведения об издании. Если целое число, то выводится в виде «5-е изд.». Если литерал, выводится как есть. Например: <<издание 13-е, существенно переработанное>>.

\listitem{editor, translator}{name}

Форма имён в этих полях зависит от того, как вы будете их использовать.
Например, по умолчанию  \kvopt{useeditor}{false} и редактор выводится в виде:

\begin{bibexample}
\item \ldots\ \slash\ под ред. А.\,Петровой\ldots
\end{bibexample}

Поэтому имена в полях \bibfield{editor}, \bibfield{editora}, \bibfield{editorb},
\bibfield{editorc} должны быть в родительном падеже. Если же \kvopt{useeditor}{true},
то редактор выводится как

\begin{bibexample}
\item \textit{Петрова А.}, ред. \ldots
\end{bibexample}

и имя должно быть в именительном падеже. То же касается переводчика и опции \opt{usetranslator}
(\ldots~/~пер. с фр. \prm{кого} И.\,Ивановой).

\fielditem{journaltitle}{literal}

В \bibtype{article} отсутствие поля \bibfield{journaltitle} означает
электронную публикацию, такую как \texttt{arXiv} и пр.  
В этом случае не выводится информация об идентифицирующем документе:
сведения об ответственности, дата, том, номер и т.\,д.

\fielditem{series}{integer/literal}

Для периодических изданий, таких как журналы, выводится в виде «3-я сер.», если
является целым числом. В противном случае, выводится как есть.
О серии см. также \secref{series}.

\fielditem{issue}{integer/literal}

Поле \enquote{Выпуск} имеет разный смысл для книг и журналов. Для книг, если является
целым числом, то выводится в форме «Вып.~\prm{номер}», если литералом~--- выводится как есть.
Для журналов и прочей периодики должно быть литералом (например, Summer, Autumn и т.\,д.).
%Подробнее см. стр.~\pageref{serinjour}.

\fielditem{shorttitle}{literal}
\label{ellipsis}

Команда \cmd{textellipsis}, так же как и символ \enquote{…} (Unicode U+2026), в поле \bibfield{shorttitle} воспринимается \biblatex'ом
как конец предложения. Это может приводить к конфликтам с трекером пунктуации.
По этой причине рекомендуется добавлять в таких случаях
команду \cmd{isdot}, превращающую
последнюю точку в точку сокращения (\texttt{dot}), а не конца предложения
(\texttt{period}). Например:

\begin{ltxcode}[escapechar=|]
@BOOK{book,
  ...
  shorttitle = {|Введение|~\textellipsis\isdot}
}
\end{ltxcode}

Или:

\begin{ltxcode}[escapechar=|]
@BOOK{book,
  ...
  shorttitle = {|Введение|~...\isdot}
}.
\end{ltxcode}

\end{fieldlist}


%\subsection{Электронные ресурсы}
%\label{sec:elres}
%
%Для электронных ресурсов в области специфических сведений содержатся вид и объём ресурса
%в виде: <<Вид (Объём)>> (см. ГОСТ 7.82-2001, п.5.5). Например:
%
%\begin{bibexample}
%\item \textellipsis~"--- Электрон, дан. (3 файла) и прогр. (2 файла).~"--- \textellipsis
%\item \textellipsis~"--- Электрон, дан. (2 файла : 70 тыс. записей).~"--- \textellipsis
%\item \textellipsis~"--- Электрон, прогр. (2 файла : 18650 байтов).~"--- \textellipsis
%\item \textellipsis~"--- Electronic text data (2 files : 1.6 Mbytes).~"--- \textellipsis
%\end{bibexample}

\subsection{Тома, книги, части, выпуски}
\label{sec:volsparts}

\begin{wraptable}{o}{.5\textwidth}
%\begin{table}[htbp]
\tablesetup
\centering
\vspace{-\baselineskip}
\caption{Обозначение частей издания\label{tab:volsparts}}
\begin{tabularx}{.4\textwidth}{@{}L{.2\textwidth}@{}V{.2\textwidth}@{}}
\toprule
\multicolumn{1}{@{}H}{Части издания} &
\multicolumn{1}{@{}H}{Поле \biblatex}  \\
\cmidrule(r){1-1}\cmidrule{2-2}
тома & volume, volumes \\
книги & book, books \\
части & part, parts \\
выпуски & issue%
%\footnote{Поле \bibfield{issue} имеет особое значение в случае журналов.
%См. п. ??? руководства по \biblatex.}
, issues \\
\bottomrule
\end{tabularx}
%\end{table}
\end{wraptable}

Издание может разделяться на физически отдельные части, которые могут называться по-разному. \biblatexgost поддерживает деление на тома, книги, части и выпуски.
Соответствующие им поля базы данных показаны в \tabref{tab:volsparts}.

Если эти поля являются целыми числами, то они выводятся в виде: \enquote{В 5 т.},
\enquote{В 4 вып.} и т.\,д. В противном случае, они выводятся как есть: \enquote{В
5-и томах (6-и кн.)} и т.\,д. Строковые значения при занесении в базу данных следует
писать со строчной буквы, \biblatexgost перейдёт на заглавные там, где требуется. Например:

\begin{lstlisting}[style=bibtex,escapechar=|]
volumes = {|в 5-и томах (6-и кн.)|}
\end{lstlisting}

По умолчанию части выводятся в порядке: том, книга, часть, выпуск. Эту последовательность
можно изменить опцией пакета \opt{volsorder}. Она представляет собой строку, состоящую из
символов \opt{v}, \opt{b}, \opt{p}, \opt{i} в любом порядке.   % и количестве.
Порядок символов определяет порядок вывода частей. Например, по умолчанию \kvopt{volsorder}{vbpi}.
Аналогичным образом можно задавать порядок для отдельной записи
с помощью поля \bibfield{volsorder}. Его значение наследуется по ссылкам \bibfield{crossref}
и \bibfield{xref}.

Несколько примеров:

\bigskip

\columnseprule=.5pt
\raggedcolumns
\begin{multicols}{2}
\begin{lstlisting}[style=bibtex,escapechar=|]
volume = {2},
book = {1},
volsorder = {vb},
\end{lstlisting}
\columnbreak
Выводится как:
\begin{bibexample}
\item \ldots\ Т. 2. Кн. 1. \ldots
\end{bibexample}
\columnbreak
\end{multicols}

\bigskip

\begin{multicols}{2}
\begin{lstlisting}[style=bibtex,escapechar=|]
book = {1},
issue = {4},
part = {2},
volsorder = {pbi},
\end{lstlisting}
\columnbreak
Выводится как:%
\begin{bibexample}
\item \ldots\ Ч. 2. Кн. 1. Вып. 4. \ldots
\end{bibexample}
\columnbreak
\end{multicols}

\bigskip
И несколько экзотический пример:
\bigskip

\begin{multicols}{2}
\begin{lstlisting}[style=bibtex,escapechar=|]
@COLLECTION{coll,
  title = {|Сказки|},
  volumes = {5},
  books = {6},
  volsorder = {vb},
  ...
}

@COLLECTION{coll3,
  crossref = {coll},
  title = {|Сказки Центральной Африки|},
  volume = {3},
  book = {|книга первая|},
  ...
}
\end{lstlisting}
\columnbreak
Выводится как:
\begin{bibexample}
\item Сказки : в 5 т., 6 кн. \ldots
\item Сказки. В 5 т., 6 кн. Т. 3. Книга первая. Сказки Центральной Африки \ldots
\end{bibexample}
\end{multicols}

\bigskip
Обратите внимание, что механизм сортировки \biblatex\unspace\slash\biber
нечувствителен к значениям опции %\opt{volsorder}
и поля \bibfield{volsorder}, поэтому при их изменении может потребоваться ручная настройка
сортировки, например, при помощи поля \bibfield{sortvolume} (см.~\secref{sec:newfields}).

\subsection{Предполагаемые и открытые даты}
\label{sec:dates}

В пакете частично реализована обработка дат в формате EDTF (см. раздел
<<Date and Time Specifications>> в документации \biblatex).  
Вывод соответствует \gostbibref{5.5.5.3}, см. примеры в
\tabref{tab:dates}.  
Прочерк в таблице означает, что спецификация не реализована "---
поскольку требования ГОСТа на этот счёт неясны.
\begin{table}[htbp]
  \tablesetup
  \centering
  \caption{Предполагаемые даты}
  \label{tab:dates}
  \begin{tabularx}{.6\linewidth}{V{.3\linewidth}X}
    \toprule
    \multicolumn{1}{H}{Значение поля \bibfield{date}} &
                                                        \multicolumn{1}{H}{Вывод в тексте} \\
    \cmidrule(r){1-1}\cmidrule{2-2}
    1997/unknown & 1997— \\
    1997/open & 1997— \\
    1942? & [1942?] \\
    1900\textasciitilde & [ок. 1900] \\
    1900?\textasciitilde & [ок. 1900?] \\
    199u  & [199\bibrangedash] \\
    18uu  & [18\bibrangedash\addnbspace\bibrangedash] \\
    % 17uu? & [17\bibrangedash\addnbspace\bibrangedash?] \\
    % 17uu? & "--- \\ 
    1999-uu & "--- \\ 
    1999-01-uu & "--- \\ 
    1999-uu-uu & "--- \\
    % 1993/2002? & [1993—2002?] \\
    % 1993/2002\textasciitilde &  [1993\bibrangedash ок. 2002] \\
    % 1993?\textasciitilde/2002? & [ок. 1993?\bibrangedash 2002?] \\
    \bottomrule
  \end{tabularx}
\end{table}

Нестандартные даты, такие как [1898 или 1899], [между 1908 и 1913],
[конец XIX "--- нач. XX в.] и пр., можно помещать в поле
\bibfield{year} в виде простой строки.


\subsection{Оформление диссертаций и авторефератов}
\label{sec:dissers}

Для оформления диссертаций и автореферетов используйте записи типа \bibtype{thesis}.  
Поле \bibfield{type} определяет тип диссертации "--- магистерская (\opt{mathesis}),
кандидатская (\opt{phdthesis}), докторская (\opt{docthesis}) "--- или тип автореферата:
кандидатская (\opt{phdautoref}), докторская (\opt{docautoref}).
При других значениях поля \bibfield{type}, оно выводится как есть, причём поле
\bibfield{science} (см.~ниже) игнорируется.

Для записей типа \bibtype{thesis} определены дополнительные поля:

\begin{fieldlist}

\fielditem{science}{literal}\noitemspace%
\fielditem{major}{literal}

Отрасль науки в виде «ист. наук», «физ.-мат. наук», «искусствоведения» и т.\,д.
Выводится как «дис. \textellipsis\ канд. ист. наук», «дис. \textellipsis\
канд. физ.-мат. наук», «дис. \textellipsis\ канд. искусствоведения» и т.\,д.

Поля эквивалентны, можно использовать любое из них, но \bibfield{major} является
устаревшим и не рекомендуется к использованию.

\fielditem{specialitycode}{literal}\noitemspace%
\fielditem{number}{literal}\noitemspace%
\fielditem{majorcode}{literal}

Код специальности в виде <<09.00.01>>, <<07.00.02>> и т.\,д.

Поля эквивалентны, можно использовать любое из них, но \bibfield{majorcode} является
устаревшим и не рекомендуется к использованию.

\fielditem{speciality}{literal}

Название специальности.  
Отделяется от кода специальности разделителем \cmd{specialitydelim}, который по умолчанию
определён как тире с пробелами:

\begin{lstlisting}[style=latex]
\newcommand*{\specialitydelim}{\addnbspace\textemdash\space}
\end{lstlisting}

\end{fieldlist}

Пример оформление диссертации:

\begin{lstlisting}[style=bibtex,escapechar=|]
@THESIS{belozerov_thesis,
  author         = {|Белозеров, Иван Валентинович|},
  title          = {|Религиозная политика Золотой Орды на Руси в XIII—XIV вв.|},
  media          = {text},
  type           = {phdthesis},
  science        = {|ист. наук|},
  specialitycode = {07.00.02},
  titleaddon     = {|защищена 22.01.02~: утв. 15.07.02|},
  location       = {|М.|},
  date           = {2002},
  pagetotal      = {215},
  addendum       = {|Библиогр.: с. 202---213.~--- 04200201565|},
  langid         = {russian},
}
\end{lstlisting}

Другие примеры можно найти в файле примеров.

\biblatex позволяет определять новые типы записей и динамически соотносить их
с уже имеющимися.
Например, если вы предпочитаете описывать докторские диссертации записями типа
\bibtype{docdisser}, то следующий код в преамбуле

\begin{lstlisting}[style=latex]
\DeclareSourcemap{
  \maps[datatype=bibtex]{
    \map{
      \step[typesource=docdisser, typetarget=thesis, final]
      \step[fieldset=type,        fieldvalue=docthesis]
    }
   }
}
\end{lstlisting}

\noindent сделает \bibtype{docdisser} эквивалентным \bibtype{thesis} с установленным полем
\kvopt{type}{docthesis}.
В частности, в \biblatex и \biblatexgost уже определены типы \bibtype{masterthesis} (с
типом \opt{mathesis}), \bibtype{phdthesis} (с типом \opt{phdthesis}), \bibtype{candthesis}
(с типом \opt{phdthesis}) и \bibtype{docthesis} (с типом \opt{docthesis}). 


\subsection{Оформление патентов}
\label{sec:patent}

Для оформления патентов определены следующие дополнительные поля (см. \gostbibref{5.4.1.5}):

\begin{fieldlist}

\fielditem{ipc}{literal}

Код Международной патентной классификации (МПК) или Международной
классификации изобретений (МКИ).

\fielditem{authorcountry}{literal}

Страна автора патента.

\fielditem{requestnumber}{literal}

Регистрационный номер заявки на патентный документ.

\fielditem{publicationdate}{date}\noitemspace%
\fielditem{publication}{literal}

Дата публикации и сведения об официальном издании, в котором опубликованы сведения о
патентном документе.

\fielditem{prioritydate}{date}\noitemspace%
\fielditem{prioritynumber}{literal}\noitemspace%
\fielditem{prioritycountry}{literal}

Сведения о конвенционном приоритете: дата подачи заявки, номер и название
страны конвенционного приоритета.
Вместо названия страны можно использовать локализованные
идентификаторы, такие как \texttt{countryussr} или \texttt{countryfr}
"--- см.~примеры.

\end{fieldlist}

Номер патента и дата подачи (поступления) заявки заносятся в поля \bibfield{number} и
\bibfield{date}.

Поле \bibfield{type} может иметь значения \opt{patent}, \opt{patreq} и \opt{invcert},
соответствующие патенту, заявке и авторскому свидетельству.

\pagebreak[2]

Для совместимости с предыдущими версиями часть полей имеет
альтернативные имена, приведённые в~\tabref{tab:patent-fields}.
Их использование не рекомендуется.

% \begin{wraptable}{r}{0pt}
\begin{table}[htb]
% \begingroup
  \tablesetup
  \centering
  \caption{Альтернативные имена полей для патентов\label{tab:patent-fields}}
  \begin{tabularx}{.6\textwidth}{@{}V{.3\textwidth}@{}V{.3\textwidth}@{}}
    \toprule
    \multicolumn{1}{@{}H}{Имя} &
                                 \multicolumn{1}{@{}H}{Альтернативное имя}  \\
    \cmidrule(r){1-1}\cmidrule{2-2}
    authorcountry & authortype \\
    requestnumber & reqnumber \\
    publicationdate & publdate \\
    prioritydate & prdate \\
    prioritynumber & prnumber \\
    prioritycountry & prcountry \\
    \bottomrule
  \end{tabularx}
% \endgroup
\end{table}
% \end{wraptable}

Пример оформление патента под заглавием (без заголовка):
\begin{lstlisting}[style=bibtex,escapechar=|]
@PATENT{patent2,
  author          = {|Тернер, Э. В.|},
  authorcountry   = {|США|},
  title           = {|Одноразовая ракета-носитель|},
  media           = {text},
  type            = {patreq},
  number          = {1095735},
  location        = {countryru},
  ipc             = {|МПК|\ensuremath{^7} B 64 G 1/00},
  holder          = {{|заявитель Спейс Системз/Лорал, инк.|}},
  credits         = {|патент. поверенный Егорова Г. Б.|},
  requestnumber   = {2000108705/28},
  date            = {2000-04-07},
  publicationdate = {2001-03-10},
  publication     = {|Бюл. № 7 (I ч.)|\midsentence},
  prioritydate    = {1999-04-09},
  prioritynumber  = {09/289, 037},
  prioritycountry = {countryus},
  pagetotal       = {|5 с.~: ил.|},
  langid          = {russian},
}
\end{lstlisting}

Тот же самый патент, оформленный под заголовком (вместо
\bibfield{type}, \bibfield{number}, \bibfield{location}, \bibfield{ipc}
используется поле \bibfield{heading}):

\begin{lstlisting}[style=bibtex,escapechar=|]
@PATENT{patent2,
  heading = {|Заявка 1095735 Рос. федерация, МПК|\ensuremath{^7} B 64 G 1/00},
  author          = {|Тернер, Э. В.|},
  authorcountry   = {|США|},
  title           = {|Одноразовая ракета-носитель|},
  media           = {text},
  holder          = {{|заявитель Спейс Системз/Лорал, инк.|}},
  credits         = {|патент. поверенный Егорова Г. Б.|},
  requestnumber   = {2000108705/28},
  date            = {2000-04-07},
  publicationdate = {2001-03-10},
  publication     = {|Бюл. № 7 (I ч.)|\midsentence},
  prioritydate    = {1999-04-09},
  prioritynumber  = {09/289, 037},
  prioritycountry = {countryus},
  pagetotal       = {|5 с.~: ил.|},
  langid          = {russian},
}
\end{lstlisting}

Другие примеры можно найти в файле примеров.

\subsection{Оформление стандартов}
\label{sec:standards}

Согласно \gostbibref{5.4.1.4},
<<При описании нормативных документов по стандартизации (стандартов и технических
условий) в области специфических сведений указывают обозначение ранее действовавшего документа,
даты введения, сроки действия объекта библиографического описания>>.
Таким образом, стандарты можно описывать как записи типа \bibtype{reference} с
заполненным полем \bibfield{specdata}. Например:

\begin{lstlisting}[style=bibtex,escapechar=|]
@REFERENCE{standard3,
  title        = {|Селитра калиевая техническая. Технические условия|},
  media        = {text},
  subtitle     = {|ГОСТ 19790-74|},
  specdata     = {|Взамен ГОСТ 1949-65 и ГОСТ 5.1138-71 ; введ. 01.07.05|},
  location     = {|М.|},
  publisher    = {|Стандартинформ|},
  year         = {2006},
  pagetotal    = {18},
  series       = {|Межгосударственный стандарт|},
  langid       = {russian},
}
\end{lstlisting}

Тот же самый ГОСТ, оформленный под заголовком (вместо \bibfield{subtitle}
используется \bibfield{heading}):

\begin{lstlisting}[style=bibtex,escapechar=|]
@REFERENCE{standard3,
  heading      = {|ГОСТ 19790-74|},
  title        = {|Селитра калиевая техническая. Технические условия|},
  media        = {text},
  specdata     = {|Взамен ГОСТ 1949-65 и ГОСТ 5.1138-71 ; введ. 01.07.05|},
  location     = {|М.|},
  publisher    = {|Стандартинформ|},
  year         = {2006},
  pagetotal    = {18},
  series       = {|Межгосударственный стандарт|},
  langid       = {russian},
}
\end{lstlisting}

Другие примеры можно найти в файле примеров.

\section{Работа с пакетом}
\label{sec:usage}

Работа со стилями \biblatexgost не отличается от работы с любыми другими стилями
\biblatex. Например, в случае использования \sty{babel} ваш
файл \latex может выглядеть примерно так:

\begin{ltxexample}[escapechar=|]
\documentclass{...}
\usepackage[french,latin,english,main=russian]{babel}
...
\usepackage[style=gost-footnote,        % |стиль цитирования и библиографии|
  language=auto,                        % |получение языка из| babel
  autolang=other,                       % |многоязычная библиография|
  ...
]{biblatex}
\addbibresource{file.bib}               % |библиографическая база данных|
...
\begin{document}

...~\cite[123]{knuth:ct:a} ...          % |цитирование|

\printbibliography                      % |печать библиографии|

\end{document}
\end{ltxexample}

Подробности можно найти в документации по \biblatex{}\fnurl{http://mirror.ctan.org/macros/latex/contrib/biblatex/doc/biblatex.pdf}. 

\subsection{Новые опции и значения опций}
\label{sec:newoptions}

Для всех стилей сохранены опции стандартных стилей \biblatex, от которых они происходят или
с которыми сходны, даже если соответствующая функциональность не предполагается ГОСТом.
Кроме этого, пакет \biblatexgost имеет
следующие дополнительные опции.

\begin{optionlist}

  \choitem[emdash]{blockpunct}{emdash, space}

  Разделитель блоков библиографического описания.  
  По умолчанию равен точке и тире с окружающими его пробелами, при значении
  \kvopt{blockpunct}{space} "--- точке с пробелом.

  Опция просто устанавливает параметр \cmd{newblockpunct}.
  Например, по умолчанию устанавливается:
  \begin{lstlisting}[style=latex]
  \renewcommand*{\newblockpunct}{%
    \addperiod\addnbspace\textemdash\space\bibsentence}.
  \end{lstlisting}
  \vspace{-\baselineskip}Если вам нужен другой разделитель, измените его с помощью \cmd{renewcommand}.    
  
\boolitem{citeurl,biburl}\noitemspace%
\boolitem{citeisbn,bibisbn}\noitemspace%
\boolitem{citedoi,bibdoi}\noitemspace%
\boolitem[\textrm{см.\,\tabref{tab:bibcite}}]{citeeprint,bibeprint}%\noitemspace%

В дополнение к опциям \opt{url}, \opt{isbn}, \opt{doi} и \opt{eprint} в \biblatexgost
определены опции
для раздельного регулирования этих параметров для цитат и библиографии. При этом опции
\opt{url}, \opt{isbn}, \opt{doi} и \opt{eprint} устанавливают одновременно соответствующие
\opt{cite\ldots} и \opt{bib\ldots} опции. Не все из этих опций имеют смысл для всех
стилей (см. ниже). По умолчанию стили, для которых опции
\opt{cite\ldots} имеют смысл, устанавливают их в \opt{false}.
Исключением является стиль \bibsty{gost-footnote}, который
устанавливает \kvopt{citeisbn}{true}. Опции \opt{bib\ldots} устанавливаются
по умолчанию в \opt{true}
полными стилями и в \opt{false}~--- минимальными. Значения по умолчанию для всех опций
представлены
в \tabref{tab:bibcite}. Прочерк в таблице означает, что опция для данного стиля
не имеет смысла.

В записях типа \bibtype{online} URL выводится всегда, независимо от значений опций
\opt{url}, \opt{biburl} и \opt{citeurl}.

\begin{table}[htbp]
\tablesetup
\centering
\caption{Значения по умолчанию для опций \opt{citeurl}, \ldots, \opt{biburl}, \ldots}
\label{tab:bibcite}
\begin{tabularx}{.8\textwidth}{@{}V{.3\textwidth}@{}V{.3\textwidth}@{}V{.2\textwidth}@{}}
\toprule
\multicolumn{1}{@{}H}{Стиль} &
\multicolumn{1}{@{}H}{Опции \opt{cite\ldots}} &
\multicolumn{1}{@{}H}{Опции \opt{bib\ldots}}  \\
\cmidrule(r){1-1}\cmidrule{2-3}%\cmidrule{3-3}
gost-inline & false & true \\
gost-footnote & false (\rmfamily{кроме} \opt{citeisbn}) & true \\
gost-numeric & --- & true \\
gost-authoryear & --- & true \\
gost-inline-min & false & false \\
gost-footnote-min & false & false \\
gost-numeric-min & --- & false \\
gost-authoryear-min & --- & false \\
\bottomrule
\end{tabularx}
\end{table}


\boolitem[true]{doublevols}

Согласно \gostbibref{7.5.1.1}, сдвоенные номера тома или выпуска приводят через
косую черту.  
В \biblatexgost это происходит по умолчанию, причём для всех языков.  
Данная опция позволяет отключить это поведение.
При \kvopt{doublevols}{false} разделитель определяется настройками языка (командой
\cmd{bibrangedash} в файлах \file{.lbx}).  
Кроме того, при \kvopt{doublevols}{true} разделитель определяется командой
\cmd{doublevolsdelim}, определённой по умолчанию как косая черта:
\begin{lstlisting}[style=latex]
  \newcommand{\doublevolsdelim}{\slash}.
\end{lstlisting}

\boolitem[false]{dropdates}

Опция определена для стилей \bibsty{gost-authoryear} и \bibsty{gost-authoryear-min}.  
Для таких стилей ГОСТ разрешает опускать дату, если она не создаёт двусмысленности,
например, если в тексте цитируется только одна работа данного автора или только один
сборник без автора с данным названием.  
Опция \opt{dropdates} включает и выключает этот механизм.  
По умолчанию даты выводятся всегда.

Опция определена для удобства.  
В настоящий момент она лишь включает/выключает опции \biblatex \opt{singletitle} и
\opt{uniquebaretitle}.

\boolitem[false]{inbookibid}

Опция определена только для стилей \bibsty{gost-inline} и \bibsty{gost-footnote}
и отключена по умолчанию. Согласно
\gostciteref[у]{8.11}, в случае последовательных ссылок на публикации, включённые в один и
тот же документ, во второй и последующих ссылках вместо повторяющихся сведений о документе
ставят слова
\enquote{Там же} или \enquote{Ibid.}:

\begin{bibexample}
\item \emph{Декарт Р.} Первоначала философии // Сочинения в двух томах. Т. 1. М.: Мысль, 1989.
С. 12~; \emph{Декарт Р.} Рассуждение о методе // Там же. С. 112
\end{bibexample}

Чтобы это обеспечить, \biblatexgost ориентируется на поле \bibfield{crossref},
считая сведения повторяющимися, если повторяется это поле. По этой причине,
для корректной работы этого механизма требуется аккуратное заполнение поля \bibfield{crossref}.
Поскольку не для всякой библиографической базы данных это верно,
такое поведение может оказаться нежелательным. Данная опция позволяет его отключить.

Обратите внимание, что поле \bibfield{crossref} доступно только в том случае, если
доступна запись, на которую оно указывает, то есть если эта запись также цитируется
в тексте или стоит в аргументе команды \cmd{nocite} (см.~документацию \biblatex).
Она также становится доступной автоматически (и включается в список литературы),
когда количество цитируемых записей с одним и тем же полем \bibfield{crossref}
превысит порог, задаваемый опцией \opt{mincrossrefs}.
По умолчанию этот параметр равен 2, но
при установке \kvopt{inbookibid}{true} \biblatexgost выставляет его равным 1,
так, чтобы запись, на которую указывает \bibfield{crossref}, всегда включалась в
библиографию, и поле \bibfield{crossref}  всегда было доступно.
Если вы изменяете опцию \opt{mincrossrefs} в настройках, убедитесь, что записи,
нужные для работы опции \opt{inbookibid}, присутствуют в вашем тексте хотя бы в
аргументе команды \cmd{nocite}.

Опция относится к записям типа \bibtype{article}, \bibtype{inbook}, \bibtype{bookinbook},
\bibtype{suppbook}, \bibtype{incollection}, \bibtype{inproceedings} и т.\,д.

\choitem[gostletter]{mergedate}{goststrict, gostletter, gostlabel}

В \biblatex эта опция регулирует способ вывода даты в библиографии для стилей
типа \bibsty{authoryear}. Стандартные стили \biblatex могут выводить дату в двух местах.
Во-первых, в скобках сразу после автора:

\begin{bibexample}
\item \emph{Хайдеггер М.} (1997). Бытие и время. \ldots
\item \emph{Хайдеггер М.} (1993a). Время и бытие. \ldots
\end{bibexample}

Во-вторых, на своём обычном месте в самой записи.
Данная опция определяет, каким образом выводится вторая дата:
\begin{enumerate}[label=\arabic*)~]
\item как обычно (\kvopt{mergedate}{false}),
\item выводит дату только для меток с добавленной буквой (\kvopt{mergedate}{minimum}),
\item выводит дату только для меток с добавленной буквой, если
она является просто годом (\kvopt{mergedate}{basic}),
\item дата вообще не выводится на своём обычном месте (\kvopt{mergedate}{maximum}),
\item то же, но поле \bibfield{issue} всё же выводится
(\kvopt{mergedate}{compact {\rmfamily или} true}).
\end{enumerate}
Подробности см. в документации \biblatex, в примерах.
\biblatexgost сохраняет все эти опции, но поскольку ГОСТ не предусматривает вывод даты
после автора, то добавляет ещё три.
Все они чувствительны к опции \opt{dropdates} и могут опускать
дату, если она не требуется для устранения неопределённости.

\begin{valuelist}

\item[gostletter]

Для предотвращения неопределённых ссылок (например, при совпадении автора и года) к году при
необходимости добавляется латинская буква. Эта функциональность не предусмотрена ГОСТом, но
является стандартной для стилей типа \bibsty{authoryear}. Год с буквой выводится на его обычном
месте:

\begin{bibexample}
\item \emph{Пушкин А.\,С.} Руслан и Людмила. \ldots{} 1963b.~--- \ldots
\end{bibexample}

\item[gostlabel]

Ссылки оформляются так же, как и в предыдущем случае, но в библиографии записи начинаются с метки,
применённой при цитировании (автор или заглавие и год в квадратных скобках), а к
году на обычном месте буква не прибавляется:

\begin{bibexample}
\item {[Пушкин, 1990c]} \emph{Пушкин А.\,С.} Сказки. \ldots{} 1990.~--- \ldots
\end{bibexample}

По существу, эта опция отличается от \kvopt{mergedate}{false} только деталями
оформления.%, но не способом вывода дат.

Обратите внимание, если вы используете поле \bibfield{shorthand}, то при
использовании этой опции вам может
потребоваться вручную настроить сортировку, например, при помощи поля \bibfield{sortkey}.

В настоящее время эта опция вряд ли необходима.
Последние версии \biblatex имеют очень гибкий механизм настройки меток для стилей типа
\bibsty{alphabetic}, с помощью которого можно обеспечить ту же функциональность.
См.~описание команды \cmd{DeclareLabelalphaTemplate} в документации \biblatex.

\item[goststrict]

Эта опция призвана обеспечить максимальное соответствие требованиям ГОСТа,
хотя при этом могут возникать неопределённые ссылки.
Никаких знаков к году не добавляется, и вам нужно самим следить за
уникальностью ссылок (\biblatexgost выводит предупреждение при обнаружении неуникальности).
Например, если вы делаете ссылки на тома многотомного издания, то вам, возможно,
понадобится указать том в \prm{postnote} или воспользоваться командами типа \cmd{volcite}.

Эта опция также устанавливает \kvopt{drodates}{true}, то
есть при ней по умолчанию год в ссылке не выводится, если он не требуется для
устранения неопределённости.

\end{valuelist}

\boolitem[true]{movenames}
\label{4names}

Опция разрешает или запрещает перемещение имён в область сведений об ответственности,
если количество имён больше трёх.

При наличии 4-х и более имён авторов, редакторов, переводчиков и пр. ГОСТ позволяет на
выбор либо выводить их полный список, либо сокращать его до
одного имени с добавлением [et al.], [и др.] и пр.
При этом сами имена выводятся не в заголовке библиографической записи,
а в сведениях об ответственности:

\begin{bibexample}
\item Хорошая книга / И. Иванов [и др.]. --- \textellipsis
\end{bibexample}

Данная опция позволяет запретить это перемещение имён в область сведений об
ответственности. При этом опция не влияет на сокращение списка имён. Последнее регулируется
независимо опциями \opt{maxnames} и \opt{minnames}, которые в \biblatexgost
по умолчанию установлены в 3 и 1,
соответственно (что означает, что при наличии более трёх имён в списке, он сокращается до
одного).
%\opt{useauthor}, \opt{useeditor} и \opt{usetranslator},
В частности, количество имён, при котором происходит перемещение (4 и более), не зависит
от значения \opt{maxnames} и вообще не
регулируется в \biblatexgost (особенности реализации см. в~\apxref{impl:4names}).
Впрочем, изменение этого количества означало бы отступление от ГОСТа.

Поскольку при перемещении имён в качестве заголовка выступает название текста, а не имена, 
такие команды как \cmd{citeauthor} работают неверно. 
Фактически, эти команды выводят список \bibfield{labelname} --- аналог заголовка в \biblatex,
который может состоять из авторов, редакторов или переводчиков, но который в данном случае 
оказывается пустым.
Если вам нужно вывести отдельно одно из этих имён, воспользуйтесь командой \cmd{citename}.

Обратите внимание, что при перемещении имён записи вида
\begin{lstlisting}[style=bibtex,escapechar=|]
author = {|Петров, А.| and others}
\end{lstlisting}
интерпретируются как записи с четырьмя и более авторами (редакторами).

\boolitem[true]{otherlangs}

Пакет переопределяет некоторые параметры локализации для других языков, помимо русского.  
В частности, например, ГОСТ требует, чтобы страницы сокращались до одной буквы, хотя в
других языках могут быть другие стандарты.  
Опция \opt{otherlangs} позволяет отключить это переопределение.  
При установке её в \texttt{false} используются стандартные настройки \biblatex. 

\choitem[true]{related}{true, false, bib, cite}

Дополнительно к имеющимся в \biblatex значениям опции \opt{related}, полностью разрешающим
или запрещающим вывод поля \bibfield{related}, в \biblatexgost определены значения, позволяющие
делать это отдельно для цитат и библиографии. По умолчанию установлено \kvopt{related}{true},
т.\,е. поле \bibfield{related} выводится и в библиографии, и в цитатах.

\choitem{sorting}{ntvy}
\label{sorting}

В \biblatexgost определена дополнительная схема сортировки \opt{ntvy}
(имя, заглавие, том\slash книга\slash часть\slash выпуск, год).
Она по умолчанию включена для стилей \bibsty{gost-inline}, \bibsty{gost-footnote}
и \bibsty{gost-numeric}.
Схема сортирует сначала по тому, затем, внутри тома, по книге, затем~--- по части, и
затем~--- по выпуску.
Кроме того, поскольку стандартные схемы сортировки \biblatex плохо учитывают многотомные
издания (неверно обрабатывают
поле \bibfield{maintitle}), то модифицированы схемы \opt{nty}, \opt{nyvt}, \opt{ynt},
\opt{anyt}, \opt{anyvt},
\opt{ydnt} и \opt{nyt}. В них также добавлена сортировка по
тому\slash книге\slash части\slash выпуску.
Схема \opt{nyt} (имя, год, заглавие) включена по умолчанию в стиле
\bibsty{gost-authoryear}.

См. также \secref{sec:volsparts} о порядке вывода томов для многотомных изданий.

\optitem[vbpi]{volsorder}{последовательность символов \opt{v}, \opt{b}, \opt{p}, \opt{i}}
\label{volsorder}

Порядок вывода томов для многотомных изданий по умолчания. Подробнее см.~\secref{sec:volsparts}.

\end{optionlist}

\subsection{Команды цитирования}
\label{sec:citecommands}

Доступные команды цитирования перечислены в следующей таблице.
%\tabref{tab:citecommands}.
Их поведение
лишь в редких случаях отличается от стандартного, описанного в документации
\biblatex.
%Цветом выделены команды, поведение которых отличается от стандартного в \biblatex.
\nopagebreak
\begingroup
\tablesetup
%\setlength\LTleft{0pt}
%\setlength\LTright{0pt}
\begin{longtable}[l]{%
    @{}V{0.25\textwidth}%
    @{}V{0.25\textwidth}%
    @{}L{0.5\textwidth}@{}}
\caption{Поддерживаемые команды цитирования\label{tab:citecommands}} \\
\toprule
\multicolumn{1}{@{}H}{Стиль} &
\multicolumn{1}{@{}H}{Команда} &
\multicolumn{1}{@{}H}{Вывод\hfill} \\
\cmidrule(r){1-1}\cmidrule(r){2-2}\cmidrule{3-3}
\endfirsthead
\captionsetup{labelformat=continued}
\caption[]{Поддерживаемые команды цитирования} \\
\toprule
\multicolumn{1}{@{}H}{Стиль} &
\multicolumn{1}{@{}H}{Команда} &
\multicolumn{1}{@{}H}{Вывод\hfill} \\
\cmidrule(r){1-1}\cmidrule(r){2-2}\cmidrule{3-3}
\endhead
\bottomrule
\endfoot
\endlastfoot
%\bibsty{Общие команды}
gost-inline & cite & Только цитата в тексте без скобок \\
  & parencite & Цитата в тексте в круглых скобках \\
  & footcite & Цитата в сноске без скобок\\
  & footcitetext & То же, но используется \cmd{footnotetext} вместо \cmd{footnote} \\
  & smartcite & \cmd{footcite} в тексте, \cmd{parencite} в сноске. Другими словами, цитата всегда
                оказывается в сноске, но в первом случае без круглых скобок,
                во втором~"--- с ними\\
  & autocite & Совпадает с \cmd{parencite} (установлена опция \kvopt{autocite}{inline})\\
\cmidrule(r){1-1}\cmidrule(r){2-2}\cmidrule{3-3}
gost-footnote & cite & Только цитата в тексте без скобок \\
  & parencite & Цитата в тексте в круглых скобках \\
  & footcite & Цитата в сноске без скобок\\
  & footcitetext & То же, но используется \cmd{footnotetext} вместо \cmd{footnote} \\
  & smartcite & \cmd{footcite} в тексте, \cmd{parencite} в сноске. Другими словами, цитата всегда
                оказывается в сноске, но в первом случае без круглых скобок,
                во втором~"--- с ними\\
  & autocite & Совпадает с \cmd{footcite} (установлена опция \kvopt{autocite}{footnote})\\
\cmidrule(r){1-1}\cmidrule(r){2-2}\cmidrule{3-3}
gost-numeric & cite & Только номер, без скобок \\
  & parencite & Номер в квадратных скобках \\
  & footcite & Номер в сноске без скобок \\
  & footcitetext & То же, но используется \cmd{footnotetext} вместо \cmd{footnote} \\
  & smartcite & \cmd{footcite} в тексте, \cmd{parencite} в сноске. Другими словами, номер всегда
               оказывается в сноске, но в первом случае без квадратных скобок,
               во втором~"--- с ними.\\
  & textcite & Номер в тексте без скобок \\
  & supercite & Номер как верхний индекс \\
  & autocite & Совпадает с \cmd{parencite} (установлена опция \kvopt{autocite}{inline})\\
\cmidrule(r){1-1}\cmidrule(r){2-2}\cmidrule{3-3}
gost-authoryear & cite & Автор (заголовок) и год в тексте без скобок \\
  & cite* & Часть цитаты без автора (заголовка) в тексте без скобок \\
  & parencite & Автор (заголовок) и год в тексте в квадратных скобках \\
  & parencite* & Часть цитаты без автора (заголовка) в тексте
                 в квадратных скобках \\
  & footcite & Автор (заголовок) и год в сноске без скобок \\
  & footcitetext & То же, но используется \cmd{footnotetext} вместо \cmd{footnote} \\
  & smartcite & \cmd{footcite} в тексте, \cmd{parencite} в сноске. Другими словами, цитата всегда
                оказывается в сноске, но в первом случае без квадратных скобок,
                во втором~"--- с ними. \\
  & textcite & Автор (заголовок, редактор, переводчик) без скобок и затем оставшаяся часть
                цитаты в квадратных скобках \\
  & autocite & Совпадает с \cmd{parencite} (установлена опция \kvopt{autocite}{inline})\\
\bottomrule
\end{longtable}
\endgroup

Кроме перечисленных поддерживаются производные команды, описанные в
 руководстве \biblatex, такие как \cmd{Cite}, \cmd{Parencite}, \dots\ \cmd{Cites},
\cmd{Footcites}, \dots\ и т.\,д.
%Исключение составляют команды типа \cmd{volcite}, которые корректно работают
%только в \bibsty{gost-numeric}.

Хотя команды \cmd{textcite} и \cmd{textcites}
% в стилях \bibsty{gost-inline}, \bibsty{gost-footnote} и \bibsty{gost-authoryear}
вряд ли найдут применение
в русском тексте, они оставлены для полноты.

%Команда \cmd{autocite} определена таким образом, чтобы\ldots

\begin{comment}----------------------------------------------------------------
\subsection{Поле \bibfield{related}}
\label{sec:related}

Поле \bibfield{related} в \biblatex~--- вместе с \bibfield{relatedtype} и
\bibfield{relatedstring}~--- служит для описания различных связей между
библиографическими записями. Связи, реализованные в \biblatexgost, перечисленны в
\tabref{tab:related}.

\begin{table}[htbp]
\tablesetup
\centering
\begin{tabularx}{.8\textwidth}{@{}V{.3\textwidth}@{}V{.3\textwidth}@{}V{.2\textwidth}@{}}
\toprule
\multicolumn{1}{@{}H}{Значения полей} &
\multicolumn{1}{@{}H}{Функция} &
\multicolumn{1}{@{}H}{Пример вывода}  \\
\cmidrule(r){1-1}\cmidrule{2-3}%\cmidrule{3-3}
gost-inline & false & true \\
\bottomrule
\end{tabularx}
\caption{Поле \bibfield{related}}
\label{tab:related}
\end{table}

(см.~файл примеров)
\end{comment}

\subsection{Формат заголовка библиографической записи}
\label{sec:headingformat}

Формат заголовка определяется командой \cmd{mkgostheading}, по умолчанию
определённой как:

\begin{lstlisting}[style=latex]
  \newcommand*{\mkgostheading}[1]{\mkbibemph{#1}}
\end{lstlisting}

В стиле \bibsty{gost-authoryear} заголовок, использованный в цитате,
форматируется отдельно:

\begin{lstlisting}[style=latex]
  \DeclareFieldFormat{citeheading}{#1}
\end{lstlisting}

Поскольку имена авторов~---
а также редакторов или переводчиков при использовании опций \opt{useeditor} и/или
\opt{usetranslator}~--- фактически выполняют роль заголовка, они по умолчанию выводятся
тем же форматом. Этот вывод можно изменить, переопределив команды
\cmd{mkbibhdnamefamily}, \cmd{mkbibhdnamegiven}, \cmd{mkbibhdnameprefix} и
\cmd{mkbibhdnamesuffix}. По умолчанию они определены так:

\begin{lstlisting}[style=latex]
  \newcommand*{\mkbibhdnamefamily}[1]{\mkgostheading{#1}}
  \newcommand*{\mkbibhdnamegiven}[1]{\mkbibhdnamefamily{#1}}
  \newcommand*{\mkbibhdnameprefix}[1]{\mkbibhdnamefamily{#1}}
  \newcommand*{\mkbibhdnamesuffix}[1]{\mkbibhdnamefamily{#1}}
\end{lstlisting}

\subsection{Другие команды}
\label{sec:other-commands}

\begin{ltxsyntax}

  \cmditem{doublevolsdelim}

  Разделитель сдвоенных номеров при \kvopt{doublevols}{true}.
  См.~\secref{sec:newoptions}.

  По умолчанию:
  \begin{lstlisting}[style=latex,belowskip=-.5\baselineskip]
    \newcommand{\doublevolsdelim}{\slash}.
  \end{lstlisting}
  Команда временно переопределяет команду \biblatex \cmd{bibrangedash}, поэтому попытка
  присваивания  
  \begin{lstlisting}[style=latex,belowskip=-.5\baselineskip]
    \renewcommand{\doublevolsdelim}{\bibrangedash}
  \end{lstlisting}
  приводит к бесконечному циклу.  
  Если вам всё же требуется такое присваивание, сделайте, например,
  \begin{lstlisting}[style=latex,belowskip=-\baselineskip]
    \let\myvoldelim\bibrangedash
    \renewcommand{\doublevolsdelim}{\myvoldelim}.
  \end{lstlisting}

  \cmditem{specialitydelim}

  Разделитель между кодом специальности и специальностью.  
  См.~\secref{sec:dissers}.

  По умолчанию:
  \begin{lstlisting}[style=latex,belowskip=-\baselineskip]
    \newcommand*{\specialitydelim}{\addnbspace\textemdash\space}.
  \end{lstlisting}

  \cmditem{subtitlepunct}

  Это стандартная команда \biblatex.  
  Она определяет разделитель между заглавием и подзаголовком и, согласно ГОСТу, равна
  двоеточию, окружённому пробелами.  
  Если вы хотите убрать пробел перед двоеточием, переопределите:
  \begin{lstlisting}[style=latex,belowskip=-.5\baselineskip]
    \renewcommand*{\subtitlepunct}{\addcolon\space}.
  \end{lstlisting}

\end{ltxsyntax}

\subsection{Описание многотомных изданий}
\label{sec:multivol}

ГОСТ определяет два способа библиографического описания отдельного тома многотомного издания:

\begin{bibexample}
\item Детская энциклопедия. В 12 т. Т. 7. Человек. \dots\
\item Человек. \dots\ (Детская энциклопедия: в 12 т., т. 7)\dots\
\end{bibexample}

В \biblatexgost реализован только первый вариант. Соответственно,
в стиле \bibsty{gost-autoryear} при отсутствии автора ссылка производится на общий
заголовок многотомного издания, а не на заголовок отдельного тома (тем более, что
последний может и отсутствовать):

\begin{bibexample}
\item{} [Детская энциклопедия, 1966b, с.\,33]
\end{bibexample}

\subsection{Библиографический список}
\label{sec:gostbibliography}

Как сказано в \secref{sec:whatfor}, пакет не предназначен для оформления библиографических
  списков и указателей.
Однако ограниченная (и экспериментальная) поддержка этой функции всё же реализована.
В пакете определяется окружение \texttt{gostbibliography}, которое можно использовать следующим
  образом:
\begin{lstlisting}[style=latex]
  \newrefcontext[sorting=ntvy]
  \printbibliography[env=gostbibliography]
\end{lstlisting}
Эта команда печатает библиографический список в соответствии с \gostbibname.
Он оформляется как простой алфавитный список без нумерации,
  независимо от того, какой стиль используется в документе в целом.
В качестве заголовка записей применяется автор(ы), т.\,е.
  невозможно использование опций \opt{useeditor} и \opt{usetranslator}.
Это связано с тем, что они требуют вывода имён редактора и переводчика одновременно
  в двух разных падежах.
В остальном учтены различия, перечисленные в \secref{sec:whatfor}.

Фактически, помимо изменений форматирования, таких как отступы и поля, окружение
\texttt{gostbibliography} содержит лишь команду
\begin{lstlisting}[style=latex]
  \toggletrue{bbx:gostbibliography}
\end{lstlisting}
Поэтому можно попытаться ограничиться только этой командой и пользоваться
обычными окружениями.  
При этом формат вывода библиографии будет соответствовать \gostbibname, а оформление
окружения (включая, например, нумерацию при \kvopt{style}{gost-numeric})
останется прежним.  
Это должно работать со всеми стилями, кроме \bibsty{gost-authoryear}, поскольку
\gostbibname изменяет вывод заголовка библиографической записи, который в этом стиле
используется для цитирования. 
Если вам всё же нужно цитирование вида «Автор-год», воспользуйтесь стилем
\bibsty{gost-alphabetic}. 

\section{Известные проблемы}
\label{sec:issues}

\begin{itemize}
\item Пакет при сортировке помещает латинские буквы перед кириллическими.
О возможных способах обхода см.:
\url{https://github.com/odomanov/biblatex-gost/wiki/}.

% Это связано с ограничениями
% модуля Unicode::Collate языка Perl, на котором написан \biber (не реализовано правило
% reorder). Поэтому, если вы хотите, чтобы в библиографии кириллические записи
% предшествовали остальным, то в качестве обходного пути можете отредактировать файл .bbl,
% создаваемый \biber'ом
% (переместить записи в начало списка \texttt{sortlist}), и затем запустить \latex.
% 
% Другой способ состоит в использовании поля \bibfield{presort} с помощью следующего кода в преамбуле:
% 
% \begin{lstlisting}[style=latex]
% \DeclareSourcemap{
%   \maps[datatype=bibtex]{
%     \map{
%       \step[fieldsource=langid, match=russian, final]
%       \step[fieldset=presort, fieldvalue={a}]
%     }
%     \map{
%       \step[fieldsource=langid, notmatch=russian, final]
%       \step[fieldset=presort, fieldvalue={z}]
%     }
%   }
% }
% \end{lstlisting}
% 
% При этом в начале библиографии помещаются записи, в которых поле \bibfield{langid} установлено в \texttt{russian}.
% Для корректной работы этого кода во всех записях исходной базы данных поле \bibfield{presort} должно быть пустым, 
% а поле \bibfield{langid}, напротив, непустым.
% Если это не так, можно попробовать следующее усложнение:
% 
% \begin{lstlisting}[style=latex]
% \DeclareSourcemap{
%   \maps[datatype=bibtex]{
%     \map{
%       \step[fieldset=langid, fieldvalue={tempruorder}]
%     }
%     \map[overwrite]{
%       \step[fieldsource=langid, match=russian, final]
%       \step[fieldsource=presort, 
%         match=\regexp{(.+)}, 
%         replace=\regexp{aa$1}]
%     }
%     \map{
%       \step[fieldsource=langid, match=russian, final]
%       \step[fieldset=presort, fieldvalue={az}]
%     }
%     \map[overwrite]{
%       \step[fieldsource=langid, notmatch=russian, final]
%       \step[fieldsource=presort, 
%         match=\regexp{(.+)}, 
%         replace=\regexp{za$1}]
%     }
%     \map{
%       \step[fieldsource=langid, notmatch=russian, final]
%       \step[fieldset=presort, fieldvalue={zz}]
%     }
%     \map{
%       \step[fieldsource=langid, match={tempruorder}, final]
%       \step[fieldset=langid, null]
%     }
%   }
% }
% \end{lstlisting}
% 
% Более тонкие стратегии сортировки могут потребовать более сложного кода.
% 
\end{itemize}

\appendix
\section*{Приложения}
\addcontentsline{toc}{section}{Приложения}

\section{Некоторые детали реализации}
\label{sec:impldet}

\begin{itemize}

\item \label{impl:4names} \biblatexgost автоматически отслеживает количество авторов,
редакторов и переводчиков, изменяя формат вывода, когда это количество превышает 3
(см.~\secref{4names}).
При этом, однако, механизм сортировки \biblatex не позволяет обнаружить, что в заголовке
теперь используются не имена, а заглавие.
Поэтому в \biblatexgost для правильной работы сортировки в записи публикаций,
имеющих более 3-х имён, в поле \bibfield{options} добавляется значение
\kvopt{useauthor}{false} (при 4-х и более авторах), \kvopt{useeditor}{false} (при 4-х и более
редакторах) и \kvopt{usetranslator}{false} (при 4-х и более переводчиках). Например:

\begin{lstlisting}[style=bibtex,escapechar=|]
@BOOK{book,
  author={|П. Первый| and |Н. Второй| and |А. Третий| and |И. Четвёртый|},
  options={useauthor=false},
  ...
}
\end{lstlisting}

Это делается автоматически при помощи функции \cmd{DeclareStyleSourcemap}%
%\footnote{В ранних версиях \biblatexgost это нужно было делать вручную.}
.
При этом, однако, используется фиксированное значение предельного количества авторов
(3), которое не зависит от опции пакета \opt{maxnames}. Соответственно, простое изменение
этой опции может привести к ошибкам сортировки. Чтобы об этом напомнить, \biblatexgost
выводит предупреждение, если при включенной опции \opt{movenames} параметры
\opt{maxbibnames} или \opt{maxcitenames} отличаются от заданных по умолчанию.

%\item Определен макрос \cmd{ifmulticitation}, принимающий значение \texttt{True}
%внутри команд \cmd{...cites} (комплексных ссылок, в терминологии ГОСТа).

\item Пакет загружает дополнительный файл локализации
\linebreak\file{russian-gost.lbx}.
%В него добавлены строки \bibfield{books}, \bibfield{parts}, \bibfield{issues} для книг, частей,
%выпусков.

\end{itemize}

\section{Последовательность макросов в драйверах}
\label{sec:macros}

Ниже приведен порядок вызова макросов~--- и, следовательно, вывода соответствующих
полей~--- в драйверах записей различных типов (см.~файл \file{gost-standard.bbx}).
Плюс в таблице означает, что макрос для записи данного типа вызывается (соответственно,
поля выводятся).
\newcommand*{\rl}{\cmidrule(r){1-1}\cmidrule(r){2-6}\cmidrule(r){7-10}\cmidrule(r){11-13}%
                  \cmidrule{14-17}}
\begingroup
\footnotesize
\tabcolsep=5.5pt
\setlength\LTleft{0pt}
\setlength\LTright{0pt}
%\noindent
\begin{longtable}{rcccccccccccccccc}\caption{Последовательность
  макросов в драйверах\label{tab:macros}} \\
\toprule
 & \begin{sideways}
inbook
\end{sideways} & \begin{sideways}
incollection
\end{sideways} & \begin{sideways}
inproceedings
\end{sideways} & \begin{sideways}
article
\end{sideways} & \begin{sideways}
article (электр.)
\end{sideways} & \begin{sideways}
book
\end{sideways} & \begin{sideways}
collection
\end{sideways} & \begin{sideways}
proceedings
\end{sideways} & \begin{sideways}
periodical
\end{sideways} & \begin{sideways}
booklet
\end{sideways} & \begin{sideways}
manual
\end{sideways} & \begin{sideways}
online
\end{sideways} & \begin{sideways}
report
\end{sideways} & \begin{sideways}
unpublished
\end{sideways} & \begin{sideways}
thesis
\end{sideways} & \begin{sideways}
misc
\end{sideways}\tabularnewline
\rl
\endfirsthead
\captionsetup{labelformat=continued}
\caption[]{Последовательность макросов в драйверах} \\
\toprule
 & \begin{sideways}
inbook
\end{sideways} & \begin{sideways}
incollection
\end{sideways} & \begin{sideways}
inproceedings
\end{sideways} & \begin{sideways}
article
\end{sideways} & \begin{sideways}
article (электр.)
\end{sideways} & \begin{sideways}
book
\end{sideways} & \begin{sideways}
collection
\end{sideways} & \begin{sideways}
proceedings
\end{sideways} & \begin{sideways}
periodical
\end{sideways} & \begin{sideways}
booklet
\end{sideways} & \begin{sideways}
manual
\end{sideways} & \begin{sideways}
online
\end{sideways} & \begin{sideways}
report
\end{sideways} & \begin{sideways}
unpublished
\end{sideways} & \begin{sideways}
thesis
\end{sideways} & \begin{sideways}
misc
\end{sideways}\tabularnewline
\rl
\endhead
\bottomrule
\endfoot
\endlastfoot
%%                                              & 1 & 2 & 3 & 4 & 5 & 6 & 7 & 8 & 9 & 0 & 1 & 2 & 3 & 4 & 5
heading                                         & + & + & + & + & + & + & + & + & + & + & + & + & + & + & + & +\tabularnewline
\rl author/translator+others                    & + & + & + & + & + &   &   &   &   &   &   &   &   &   &   & \tabularnewline
author/editor+others/translator+others          &   &   &   &   &   & + &   &   &   &   &   &   &   &   &   & \tabularnewline
editor+others                                   &   &   &   &   &   &   & + & + &   &   &   &   &   &   &   & \tabularnewline
author                                          &   &   &   &   &   &   &   &   &   &   &   &   & + & + & + & \tabularnewline
editor                                          &   &   &   &   &   &   &   &   & + &   &   &   &   &   &   & \tabularnewline
author/editor                                   &   &   &   &   &   &   &   &   &   & + & + & + &   &   &   & +\tabularnewline
\rl maintitle+volumes+parts+title               &   &   &   &   &   & + & + & + &   &   &   &   &   &   &   & \tabularnewline
title                                           & + & + & + & + & + &   &   &   & + & + & + & + & + & + & + & +\tabularnewline
type                                            &   &   &   &   &   &   &   &   &   & + & + &   & + &   &   & +\tabularnewline
type+speciality/major                           &   &   &   &   &   &   &   &   &   &   &   &   &   &   & + &  \tabularnewline
event+venue+date                                &   &   &   &   &   &   &   & + &   &   &   &   &   &   &   & \tabularnewline
translation                                     & + & + & + & + & + & + & + & + & + & + & + &   & + & + &   & +\tabularnewline%not in patent
\rl /\hfill byauthor                            & + & + & + & + & + & + &   &   &   & + & + & + & + & + & +{*} & +\tabularnewline
organization                                    &   &   &   &   &   &   &   &   &   &   & + & + &   &   &   & \tabularnewline
institution                                     &   &   &   &   &   &   &   &   &   &   &   &   & + &   &   & \tabularnewline
credits                                         & + & + & + & + & + & + & + & + & + & + & + & + & + & + & + & +\tabularnewline
byeditor                                        &   &   &   &   &   & + & + & + & + & + & + & + &   &   &   & \tabularnewline
bytranslator+others                             & + & + & + & + & + & + & + & + &   &   &   & + &   &   &   & \tabularnewline
\rl //\hfill maintitle+volumes+parts+booktitle  & + & + & + &   &   &   &   &   &   &   &   &   &   &   &   & \tabularnewline
event+venue+date                                &   &   & + &   &   &   &   &   &   &   &   &   &   &   &   & \tabularnewline
book:translation                                & + & + & + &   &   &   &   &   &   &   &   &   &   &   &   & \tabularnewline%not in patent
/\hfill book:byauthor                           & + &   &   &   &   &   &   &   &   &   &   &   &   &   &   & \tabularnewline
book:credits                                    & + & + & + &   &   &   &   &   &   &   &   &   &   &   &   & \tabularnewline
book:byeditor                                   & + & + & + &   &   &   &   &   &   &   &   &   &   &   &   & \tabularnewline
book:bytranslator+others                        & + & + & + &   &   &   &   &   &   &   &   &   &   &   &   & \tabularnewline
journal                                         &   &   &   & + &   &   &   &   &   &   &   &   &   &   &   & \tabularnewline
jour:credits                                    &   &   &   & + &   &   &   &   &   &   &   &   &   &   &   & \tabularnewline
byeditor                                        &   &   &   & + &   &   &   &   &   &   &   &   &   &   &   & \tabularnewline
\rl edition                                     & + & + &   &   &   & + & + &   &   &   & + &   &   &   &   & \tabularnewline
/\hfill editioncredits                          & + & + &   &   &   & + & + &   &   &   & + &   &   &   &   & \tabularnewline
version                                         &   &   &   &   &   &   &   &   &   &   & + & + & + &   &   & +\tabularnewline
\rl specdata                                    & + & + & + & + & + & + & + & + & + & + & + & + & + & + & + & +\tabularnewline
\rl organization                                &   &   & + &   &   &   &   & + &   &   &   &   &   &   &   & \tabularnewline
publisher+location+date                         & + & + & + &   &   & + & + & + &   &   & + &   &   &   &   & \tabularnewline
institution+location+date                       &   &   &   &   &   &   &   &   &   &   &   &   &   &   & + & \tabularnewline
organization+location+date                      &   &   &   &   &   &   &   &   &   &   &   &   &   &   &   & +\tabularnewline
location+date                                   &   &   &   &   &   &   &   &   &   & + &   &   & + & + &   & \tabularnewline
date                                            &   &   &   &   &   &   &   &   &   &   &   & + &   &   &   & \tabularnewline
\rl location                                    &   &   &   & + &   &   &   &   & + &   &   &   &   &   &   & \tabularnewline
jour:date                                       &   &   &   & + & + &   &   &   & + &   &   &   &   &   &   & \tabularnewline
jour:volume+parts+issuetitle                    &   &   &   & + &   &   &   &   & + &   &   &   &   &   &   & \tabularnewline
\rl chapter+pages                               & + & + & + &   &   & + & + & + &   & + & + &   & + &   & + & \tabularnewline
pages                                           &   &   &   & + &   &   &   &   &   &   &   &   &   &   &   & \tabularnewline
pagetotal                                       &   &   &   &   &   & + & + & + &   & + & + &   & + &   & + & \tabularnewline
\rl update, systemreq                           &   &   &   &   &   &   &   &   &   &   &   & + &   &   &   & \tabularnewline
\rl series                                      &   &   &   & + &   &   &   &   & + &   &   &   &   &   &   & \tabularnewline
series+number                                   & + & + & + &   &   & + & + & + &   &   &   &   &   &   &   & \tabularnewline
\rl isbn/issn/isrn                              & + & + & + & + &   & + & + & + & + &   & + &   & + & + & + & \tabularnewline
number                                          &   &   &   &   &   &   &   &   &   &   &   &   & + &   &   & \tabularnewline
doi+eprint+url+note                             & + & + & + & + & + & + & + & + & + & + & + & + & + &   & + & +\tabularnewline
url+urldate+note                                &   &   &   &   &   &   &   &   &   &   &   &   &   & + &   & \tabularnewline
howpublished                                    &   &   &   &   &   &   &   &   &   & + &   &   &   & + &   & +\tabularnewline
addendum+pubstate                               & + & + & + & + & + & + & + & + & + & + & + & + & + & + & + & +\tabularnewline
pageref                                         & + & + & + & + & + & + & + & + & + & + & + & + & + & + & + & +\tabularnewline
\bottomrule
\end{longtable}
\endgroup

\section{История изменений}
\label{apx:changelog}

\begin{changelog}

  \begin{release}{1.16}{23-08-2017}
  \item \bibfield{origlanguage} и \bibfield{bookoriglanguage} теперь могут быть списками
    (требуется \biblatex~3.8).
  \item Добавлена настройка вывода страниц для галицийского языка.  
    См. описание опции \opt{otherlangs}.
  \end{release}
  
  \begin{release}{1.15}{27-06-2017}
  \item Добавлен пробел перед двоеточием "--- разделителем заголовка и
    подзаголовка\see{sec:other-commands} 
  \item Исправлены ошибки вывода редакторов/переводчиков при выводе \texttt{related}. 
  \item Исправлена ошибка пунктуации при печати диссертаций.
  \end{release}

  \begin{release}{1.14}{15-04-2017}
  \item Исправлен вывод разделителя (тире) в списках \texttt{biblist}.
  \item Добавлен везде, где можно, формат \texttt{titlecase} при выводе названий. 
  \item Исправлена библиография для ГОСТ 7.1-2003\see{sec:gostbibliography}
  \item Страницы сокращаются до одной буквы, как требуется по \gostbibref{5.6.2.1, 7.4.1}.  
    Регулируется опцией \opt{otherlangs}\see{sec:newoptions}
  \item Внутренние изменения макросов вывода томов, номеров, выпусков.
  \end{release}

  \begin{release}{1.13}{16-02-2017}
  \item Требуемая версия \biblatex повышена до 3.5.
  \item Исправлена ошибка поддержки поля \bibfield{major} (не поддерживалось).
  \item Небольшие изменения в файле примеров.
  \end{release}

  \begin{release}{1.12}{05-02-2017}
  \item Определена опция \opt{doublevols} и команда
    \cmd{doublevolsdelim}\see{sec:newoptions}
  \item Определены поля \bibfield{science}, \bibfield{specialitycode},
    \bibfield{speciality}, \bibfield{number} для записей типа \bibtype{thesis}.  
    Поля \bibfield{major}, \bibfield{majorcode} объявлены устаревшими\see{sec:dissers}
  \item Вернул изменения предыдущей версии, касающиеся заполнении поля \bibfield{major}.  
    Не все отрасли науки требуют слова «наук» (например, не требуют архитектура,
    искусствоведения, культурология)\see{sec:dissers}
  \end{release}

  \begin{release}{1.11a}{28-01-2017}
  \item При заполнении поля \bibfield{major} в описании диссертаций теперь нужно писать
    «экон.», «физ.-мат.» и пр. вместо «экон. наук», «физ.-мат. наук» и пр. (в версии 1.11
    добавил это в документацию, но забыл добавить в код)\see{sec:dissers}
  \end{release}

  \begin{release}{1.11}{27-01-2017}
  \item Определены типы записей \bibtype{candthesis} и
    \bibtype{docthesis}.\see{sec:dissers}
  \item Добавлена редакторская роль \kvopt{editortype}{\{editorcollaborator\}}.\newline  
    Печатается в виде: «при ред. уч.».\see{sec:newfields}
  \item Добавлена опция \opt{blockpunct}.\see{sec:newoptions}
  \item Добавлен вывод поля \bibfield{credits} в \bibtype{thesis}.
  \item В записях типа \bibtype{thesis}, если значение поля \bibfield{type} не совпадает с
    заранее определёнными, то выводится как есть.\see{sec:dissers}
  \item Исправлена ошибка: добавлена точка после заголовка.
  \item Исправлена ошибка: не работала опция \kvopt{movenames}{false}.
  \item Заменил \texttt{build.cmd} на \texttt{build.pl}.
  \end{release}

  \begin{release}{1.10}{18-09-2016}
  \item Исправлено сокращение <<док.>> на <<д-ра>>.
  \end{release}

  \begin{release}{1.9}{12-09-2016}
  \item Удалён обход несовместимостей с \biblatex~3.5.
  \item Переписана обработка опций \opt{singletitle}, \opt{uniquebaretitle}.  
    Добавлена опция \opt{dropdates}.\see{sec:newoptions}
  \item Поправлено копирование в \texttt{build.cmd}.
  \item Изменён макрос \texttt{printdate} (расширен).
  \item В примерах добавлена сортировка русских текстов в начало библиографии.
  \item Переключатели \texttt{cbx:parens}, \texttt{cbx:loccit} изменены на
    \texttt{cbx:gost:parens}, \texttt{cbx:gost:loccit}.
  \item Исправлена ошибка вывода предполагаемых и открытых дат.
  \end{release}

  \begin{release}{1.8}{30-08-2016}
  \item Установлено \kvopt{alldates}{short}, \kvopt{eventdate}{comp}
    "--- для совместимости с \biblatex~3.5.
  \item Внутри программы поменял местами основные и альтернативные
    имена полей для патентов.  
    Для внешнего пользователя ничего не изменилось.  
  \item Добавлен обход несовместимостей с \biblatex~3.5.
  \item Исправлены ошибки вывода полей типа тома/номера/и т.\,д.
  \item Исправлены разные ошибки.
  \item Расширен файл примеров.  
  \item Изменения для совместимости с \biblatex~3.5.
  \item Добавлена обработка предполагаемых и открытых
    дат.\see{sec:dates}
  \end{release}

  \begin{release}{1.7}{18-07-2016}
  \item Добавлены альтернативные имена полей для патентов.\see{sec:patent}
  \item Исправлена ошибка в выводе \texttt{bookcredits}.
  \item Исходный код перенесён на \texttt{github}. Добавлены ссылки на
    него.\see{sec:int}
  \end{release}

  \begin{release}{1.6}{05-04-2016}
  \item Исправлены ошибки в модели данных (не влияли на вывод).
  \item Исправлена ошибка при выводе повторных (сокращённых) ссылок в
    \texttt{gost-inline/footnote}.
  \item Исправлен макрос \texttt{headingname:family-given} для
    работы с \biblatex~3.4.
  \end{release}

  \begin{release}{1.5a}{17-03-2016}
  \item Исправлена ошибка вывода имён в \texttt{gost-authoryear}
  \end{release}

  \begin{release}{1.5}{14-03-2016}
  \item Исправлен вывод имён, используемых как заголовок (в связи с
    изменением \cmd{DeclareNameFormat} в \biblatex 3.3.  
    В связи с этим изменились команды формата имён
    в заголовках.\see{sec:headingformat}
  \item Тип полей \bibfield{number}, \bibfield{book}, \bibfield{part},
    \bibfield{volume} и \bibfield{issue} изменён на \texttt{range}.
  \item Опция \texttt{firtsinits} заменена на \texttt{giveinits}
    (deprecated в \biblatex 3.3). 
  \end{release}

\begin{release}{1.4}{04-02-2016}
\item Исправлен вывод \bibtype{article} для электронных публикаций (добавлен
  вывод \bibfield{date} и \bibfield{specdata}).
\item Внутреннее имя списка сокращений изменено на \texttt{shorthand} (было
  \texttt{shorthands}). Это изменения в \texttt{biblatex}.
\item Добавлена вики с советами на SourceForge:
  \url{https://sourceforge.net/p/biblatexgost/wiki/}
\end{release}

\begin{release}{1.3}{02-05-2015}
\item Опять исправлена ошибка при выводе даты для записей типа \bibtype{article}.
\item Добавлен вывод полей \bibfield{doi} и \bibfield{eprint} в записи типа 
  \bibtype{online}.
\end{release}

\begin{release}{1.2}{01-05-2015}
\item Исправлена ошибка при выводе даты для записей типа \bibtype{article}.
\item Небольшое добавление в документации (в описании опции \texttt{movenames}).
\end{release}

\begin{release}{1.1}{28-11-2014}
\item При \kvopt{movenames}{true} (установлено по умолчанию) `\texttt{and others}' 
  в поле \bibfield{author/editor}
  означает, что число авторов/редакторов больше 3-х\see{sec:newoptions}
\item \bibsty{gost-authoryear}: исправлен вывод скобок в командах типа \cmd{parensite}.
\end{release}

\begin{release}{1.0}{15-02-2014}
\item Первая официальная версия.
\item Исправлен вывод статей без \bibfield{journaltitle} (электронные публиации). 
\item Добавлен вывод информации о переводе в сведениях, относящихся к заглавию. 
\item Исправлено несколько мелких ошибок.
\end{release}

\begin{release}{0.9.2}{30-11-2013}
\item Добавлено поле \bibfield{journalcredits}\see{sec:newfields}
\item Добавлено окружение \texttt{gostbibliography} для печати библиографических
  списков и указателей\see{sec:gostbibliography}
\item Удалён патч для настройки языка цитат, поскольку эта функция поддерживается
  в \biblatex, начиная с версии 2.8a.
\end{release}

\begin{release}{0.9.1}{4-11-2013}
\item Переход на \biblatex~2.8 и \biber~1.8.\see{sec:install}
\item Исправлены ошибки несовместимости с \biblatex~2.8.
\item Исправлено несколько ошибок пунктуации.
\end{release}

\begin{release}{0.9}{17-07-2013}
\item Переход на \biblatex~2.7 и \biber~1.7.\see{sec:install}
\item Переработан механизм \opt{bookibid}, он теперь чувствителен к трекерам
  \biblatex, работает для \bibtype{article}, были исправлены ошибки, например,
  отсутствие гиперссылок в некоторых случаях.\see{sec:newoptions}
\item Обработка поля \bibfield{related} приведена в соответствие с \biblatex,
  в связи с чем изменилось оформление.
  Кроме того, по умолчанию опция \opt{related} установлена в \opt{true}.\see{sec:newoptions}
\item Опция \opt{labelyear} заменена на \opt{labeldate}, в связи с изменениями
  в \biblatex.
\item Обновлён файл локализации.
\item Обновлён и пополнен файл примеров.
\item Исправлена ошибка несовместимости с \texttt{polyglossia}.
\item Исправлены разные ошибки.
\end{release}

\begin{release}{0.8}{03-04-2013}
\item The package status has been changed from \emph{author-maintained} to \emph{maintained}
%\item Переход на \biblatex~2.4 и \biber~1.4\see{sec:install}
\item Добавлены стили \bibsty{gost-alphabetic} и \bibsty{gost-alphabetic-min}\see{sec:styles}
\item Добавлено поле заголовка \bibfield{heading}\see{sec:newfields}
\item Добавлено оформление авторефератов диссертаций\see{sec:dissers}
\item Добавлена документация об оформлении стандартов\see{sec:standards}
\item Исправление: URL в \bibtype{online} теперь выводится всегда
\item Исправлены ошибки при переключении языков (проверка наличия \texttt{babel} и пр.)
\item Исправлены другие ошибки
\end{release}

\begin{release}{0.7.1}{15-01-2013}
\item Переход на \biblatex~2.5 и \biber~1.5\see{sec:install}
\item Небольшие правки в связи с изменениями в \biblatex~2.5
\end{release}

\begin{release}{0.7}{07-12-2012}
\item Переход на \biblatex~2.4 и \biber~1.4\see{sec:install}
\item Добавлена обработка поля \bibfield{related}%\see{sec:related}
\item Добавлены опции \kvopt{related}{bib/cite}\see{sec:newoptions}
%\item Добавлен тип записи \bibtype{doctorsthesis} для докторских диссертаций\see{sec:dissers}
%\item \bibtype{phdthesis} теперь соответствует кандидатской диссертации\see{sec:dissers}
\item Оформление диссертаций приведено в соответствие с ГОСТом\see{sec:dissers}
\item Оформление патентов приведено в соответствие с ГОСТом\see{sec:patent}
\item Добавлены поля \bibfield{update}, \bibfield{systemreq} в записях типа
  \bibtype{online}\see{sec:newfields}
\item Добавлено поле \bibfield{editioncredits}\see{sec:newfields}
\item В \bibtype{article} отсутствие поля \bibfield{journaltitle} означает электронную
  публикацию\\ (не выводится информация об идентифицирующем документе:
  сведения об ответственности, дата, том, номер и т.\,д.)
%\item Добавлен раздел документации о проблеме сортировки кириллических\\ записей\see{sec:issues}
\item Многочисленные изменения в файле локализации
\item Улучшена конкатенация строк при совпадении имён в сведениях об ответствености
\item Добавлен файл примеров \texttt{biblatex-gost-examples.pdf}
\item Небольшие изменения в драйверах (наведение порядка)
\item Добавлено предупреждение о неуникальных цитатах в режиме\\ \kvopt{mergedate}{goststrict}
\item Добавлено предупреждение об изменении \opt{maxbibnames} или \opt{maxcitenames}
%\item Добавлены строки локализации \texttt{bycompiler}, \texttt{bygecompiler}
\item Исправлена ошибка вывода редакторов в минимальных стилях
\item Исправлена обработка поля \bibfield{eprint}
\item Исправлены другие ошибки
\end{release}

\begin{release}{0.6}{20-08-2012}% up to rev.283
\item Переход на \biblatex~2.1 и \biber~1.1\see{sec:install}
\item Добавлены <<минимальные>> стили\see{sec:styles}
\item Добавлены опции \opt{cite\ldots}, \opt{bib\ldots} для раздельного управления
  выводом\\ \opt{url}, \opt{isbn}, \opt{doi} и \opt{eprint}
  в цитатах и библиографии\see{sec:newoptions}
\item Добавлены поля \bibfield{book\ldots} для раздельного описания
  публикаций в книгах\see{sec:newfields}
\item Добавлены поле и опция \bibfield{volsorder}\see{sec:newfields}
\item Добавлено поле \bibfield{sortvolume}\see{sec:newfields}
\item Поле \bibfield{material} (общее обозначение материала) переименовано в \bibfield{media}\see{sec:newfields}
\item Установлено по умолчанию \kvopt{inbookibid}{false}
\item Установлено по умолчанию \kvopt{firstinits}{true}
\item Изменён перевод некоторых строк файла локализаци
\item Исправлена ошибка вывода опции \kvopt{mergedate}{false}%\see{sec:newoptions}
\item Исправлены другие ошибки
\end{release}

\begin{release}{0.5}{1-07-2012}% up to rev. 197
\item Добавлено поле \bibfield{material} (общее обозначение материала)\see{sec:newfields}
\item Восстановлены стандартные значения опции \opt{mergedate}\see{sec:newoptions}
\item Добавлена опция \kvopt{mergedate}{gostlabel}\see{sec:newoptions}
\item Добавлена обработка опции \opt{singletitle}\see{sec:newoptions}
\item Добавлено сокращение <<Там же>>/<<Ibid.>> для  последовательных ссылок\\ на публикации,
  включённые в один и тот же документ. Регулируется опцией \opt{inbookibid}\see{sec:newoptions}
\item Исправлены отклонения от ГОСТа в драйверах \bibtype{booklet}, \bibtype{manual},
  \bibtype{misc},\\ \bibtype{online}, \bibtype{patent}, \bibtype{report},
  \bibtype{thesis}, \bibtype{unpublished}
%\item Исправлена ошибка, приводившая к неопределённости для публикаций без авторов в
%  \bibsty{gost-autoryear}%\see{sec:}
%\item Исправлена ошибка: нечисловые значения
%  \bibfield{edition}, \bibfield{volume}, \bibfield{book} и т.\,д. выводились со
%  строчной буквы
%\item Удалён вывод тома и пр. из цитат в \bibsty{gost-autoryear}%\see{sec:}
\item Исправлены ошибки
\end{release}

\begin{release}{0.4}{03-06-2012}
\item Переход на \biblatex\,2.0
\item Изменено соответствие терминов ГОСТа и полей \biblatex\see{sec:gost-biblatex}
\item Добавлено автоматическое переопределение полей (теперь не нужна настройка \file{biber.conf})%\see{bibermap}
\item Добавлена автоматическая обработка длинных списков имён\see{4names}
\item Добавлено новая опция \bibfield{movenames}\see{4names}
\item Добавлено новое поле \bibfield{specdata}\see{sec:newfields}
\item В \bibsty{gost-authoryear} значение по умолчанию опции \opt{loccittracker} изменено на
    \opt{constrict}\see{sec:styles}
\item Существенно переработана документация
\item Исправлены ошибки обработки опции \opt{loccittracker}
\item Исправлена ошибка вывода редактора в \bibtype{inbook}, \bibtype{incollection}, \ldots
\item Исправлена ошибка вывода поля \bibfield{shorttitle}
\item Исправлена ошибка вывода поля \bibfield{book}
\item Исправлена ошибка вывода серии в \bibsty{gost-inline}
\item Исправлена ошибка вывода серии в \bibtype{article} и \bibtype{periodical}
\item Исправлены другие ошибки
\end{release}

\begin{release}{0.3.2}{29-04-2012}
\item Исправлена ошибка опции \kvopt{citetracker}{constrict} в \bibsty{gost-inline},
\bibsty{gost-footnote}. Опция установлена по умолчанию.
\item Исправлено: не выводилось полное число томов, частей и т.\,д.
\item Исправлено: в цитатах \bibsty{gost-numeric} не переключался язык.
\end{release}

\begin{release}{0.3.1}{21-04-2012}
\item Исправлены ошибки обработки опции dashed
\end{release}

\begin{release}{0.3}{18-04-2012}
\item Добавлена обработка длинных списков имён\see{4names}
\item Добавлена схема сортировки \opt{ntvy}\see{sorting}
\item Изменены некоторые стандартные схемы сортировки\see{sorting}
\item Поля \bibfield{volume}, \bibfield{book}, \bibfield{part}, \bibfield{issue}
теперь могут быть нечисловыми\see{sec:volsparts}
\item Добавлены макросы настройки формата имён заголовков\see{sec:impldet}
\item Восстановлена опция \opt{dashed} для \bibsty{gost-inline}, \bibsty{gost-footnote}
\item Исправлены ошибки в формате вывода авторов, редакторов, переводчиков
(в частности, для французского языка)
\item Исправлены ошибки сортировки многотомных изданий
\item Исправлен перевод терминов \bibfield{annotator}, \bibfield{withannotator}
\item Исправлены другие ошибки
\end{release}

\begin{release}{0.2}{12-02-2012}
\item Rearranged citation commands\see{sec:citecommands}
\item \texttt{gost-intext} renamed \texttt{gost-inline}\see{sec:styles}
\item Introduced \cmd{ifmulticitation} macro\see{sec:impldet}
\item Changed documentation
\end{release}

\begin{release}{0.1}{03-02-2012}
\item Initial
\end{release}

\end{changelog}

\end{document}

%%% Local Variables:
%%% mode: latex
%%% TeX-master: t
%%% End:
